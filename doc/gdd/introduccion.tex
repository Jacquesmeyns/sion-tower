\paragraph{}
Este es el documento de diseño de \juego. El videojuego para PC que 
ejemplifica todos los contenidos de \wiki, la wiki en español sobre desarrollo de videojuegos
en 3D utilizando Ogre como motor de renderizado. Este escrito tiene como
objetivo principal plasmar los elementos que debe incluir \juego y servir
de carta de presentación en caso de buscar colaboradores en un futuro.

%\paragraph{}
%Se darán detalles sobre trasfondo, objetivos, jugabilidad, mecánicas etc.
%Finalmente se elaborará una lista de recursos artísticos necesarios: modelos,
%texturas, efectos de partículas, sonido, animaciones...

\subsection{Concepto del juego}
\label{sec:int-concepto}

\paragraph{}
\juego es un videojuego en el que controlamos a un pequeño mago iniciado
que permanece a cargo de una torre sagrada mientras sus compañeros y 
maestros han acudido a un celebrar un rito. Durante su guardia, la torre
se ve asaltada por criaturas malignas y nuestro pequeño mago debe detener
la invasión a toda costa utilizando sus limitadas habilidades.

\subsection{Características principales}
\label{sec:int-caracteristicas}

%\paragraph{}
%Hemos realizado ciertas pinceladas sobre el juego pero a continuación
%mencionaremos las características principales de \juego. En secciones
%posteriores se profundizará con detalle en todos los puntos:

\paragraph{}
El juego se basa en los siguientes pilares:

\begin{itemize}
    \item \textbf{Planteamiento sencillo}: la historia mencionada es muy
    simple, una mera excusa para el desarrollo del juego pero lo suficientemente
    explícita para que el \jugador sienta que tiene un objetivo.
    \item \textbf{Táctica}: detener a las oleadas de enemigos debe ser
    imposible si se comienza a atacar indiscriminadamente. La gestión de
    nuestras limitadas capacidades de forma inteligente será imprescindible.
    \item \textbf{Dinamismo}: al contrario que algunos juegos de estrategia,
    \juego debe ser dinámico y provocar una sensación de tensión en el
    \jugador.
    \item \textbf{Ampliación}: \juego debe ser ampliable con nuevos niveles
    y enemigos de forma sencilla. El motor será todo lo independiente posible
    del contenido. De esta forma los artistas podrán generar nuevos niveles,
    habilidades o enemigos.
\end{itemize}

\subsection{Género}
\label{sec:int-genero}

\paragraph{}
\juego supone una unión de varios géneros. A continuación se listan los géneros
de los que toma elementos y sus motivos:

\begin{itemize}
	\item \textbf{Tower Defense}: un subgénero de la estrategia basado
	en detener oleadas de enemigos colocando de forma estratégica obstáculos y trampas.
        En \juego manejamos los mismos elementos aunque aparece la figura
	de un protagonista y no es una \emph{`mano invisible'} la que gestiona la acción.
	
	\item \textbf{Acción en tercera persona}: juegos dinámicos y directos
	en el que el \jugador experimenta una descarga de adrenalina.
	La cámara suele situarse cerca del personaje. En \juego tenemos el
	componente de la acción aunque la cámara se situará de forma que podamos
	ver gran parte de la escena.
\end{itemize}

\subsection{Propósito y público objetivo}
\label{sec:int-publico}

\paragraph{}
El principal objetivo de \juego es ofrecer a los lectores de \wiki un ejemplo
real de videojuego en 3D desarrollado utilizando el engine de renderizado
Ogre. Es un complemento dentro del contenido didáctico del wiki. No obstante,
debe ser un producto jugable y divertido. Su interés no sólo debe radicar
en el código fuente y su proceso de desarrollo sino en el propio juego
y sus mecánicas.

\paragraph{}
\juego está dirigido a jugadores de un amplio rango de edades con un tiempo
limitado que dedicar al ocio electrónico. Por ello, se apuesta por un sistema
de partidas cortas y recompensas rápidas. La historia es sencilla, lo que
permite poder jugar de forma esporádica.

\subsection{Jugabilidad}
\label{sec:int-jugabilidad}

\paragraph{}
Cada nivel de \juego ofrece un piso de la torre sagrada que está siendo
asaltada por enemigos. Tenemos que impedir que las bestias lleguen a un
determinado punto del nivel. Para ello nos valdremos de los siguientes elementos:

\begin{itemize}
    \item \textbf{Movilidad}: al contrario que en otros Tower Defense, en
    \juego controlamos un personaje. Nos desplazaremos por el escenario
    atendiendo los focos de peligro que consideremos oportunos.
    \item \textbf{Trampas y obstáculos}: podemos dirigir al personaje a
    un punto del nivel y colocar una trampa u obstáculo si tenemos
    suficientes recursos. Esto eliminará o ralentizará a los enemigos.
    \item \textbf{Hechizos}: el combate directo utilizando hechizos también
    es una opción.
    \item \textbf{Puntos fuertes y débiles}: cada enemigo tendrá puntos
    fuertes y débiles. La mejor forma de hacerles frente es utilizar la
    herramienta adecuada en cada momento.
    \item \textbf{Mejoras}: el \jugador deberá recibir recompensas cada
    poco tiempo para que sienta que progresa en el juego. Tendremos habilidades
    desbloqueables.
\end{itemize}

\subsection{Estilo visual}
\label{sec:int-estilo}

\paragraph{}
\juego tendrá un estilo sencillo, no demasiado detallista para encajar con
su carácter amigable y accesible. El estilo visual que más encaja con este
concepto es el de dibujo animado o cómic. Los personajes serán caricaturescos,
los colores serán vivos y las texturas muy simples. Se estudiará aplicar un
efecto \emph{`cel-shading'} para reforzar esta idea.

\subsection{Alcance}
\label{sec:int-alcance}

\paragraph{}
El objetivo principal es desarrollar un sistema de juego sólido al que podamos
introducirle contenidos sin dificultad. En primera instancia se desarrollará
un pack de contenidos básicos que será ampliado en un futuro.

