% Importaciones y paquetes %%%%%%%%%%%%%%%%%%%%%%%%%%%%%%%%%%%%%%

% Codificación
\usepackage[utf8]{inputenc}
\usepackage[spanish,activeacute]{babel}

% Paquetes extras
\usepackage{listings} % Trozos de código
\usepackage{graphicx} % Imágenes
\usepackage{fancyhdr} % Cabeceras
\usepackage{lscape}   % Apaisado
\usepackage{hyperref} % Enlaces
\usepackage{float} % para que H funcione en figure
\usepackage[left=2.5cm,top=2.5cm,right=2cm,bottom=2.5cm]{geometry}

% Configuración  %%%%%%%%%%%%%%%%%%%%%%%%%%%%%%%%%%%%%%%%%%%%%%%%%

\lstset{language=C++,
	showstringspaces=true}

% Figura centrada con \figura{proporcion}{ruta}{caption}{label}
% Ejemplo de uso: \figura{0.5}{img.png}{Boceto}{figura1}
% El argumento proporción es opcional y por defecto es el ancho del texto
%
%                 nargs   defaults
\newcommand{\figura}[4][1]{
\begin{figure}[H!] % Aparecerá justo donde está el código
	\centering
	     \includegraphics[width=#1\textwidth]{#2}
	\caption{#3}
	\label{#4}
\end{figure}
}

% Figura centrada y rotada 90º a la derecha
% con \figura{proporcion}{ruta}{caption}{label}
% Ejemplo de uso: \figura{0.5}{img.png}{Boceto}{figura1}
% El argumento proporción es opcional y por defecto es el ancho del texto
%
%                 nargs   defaults
\newcommand{\figuraApaisada}[4][1]{
\begin{figure}[H!] % Aparecerá justo donde está el código
	\centering
	     \includegraphics[width=#1\textwidth,angle=90]{#2}
	\caption{#3}
	\label{#4}
\end{figure}
}


