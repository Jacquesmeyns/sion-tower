\section*{Concurso Universitario de Software Libre}

El proyecto \wiki\ y \juego\ ha participado en el V Concurso Universitario
de Software Libre en las categorías de \textit{Comunidad} y \textit{Educación y ocio}.
Se trata de un concurso de software, hardware y documentación libre a nivel
nacional en el que pueden participar grupos de hasta tres estudiantes universitarios,
de bachiller o de ciclos superiores. Se valora el desarrollo del proyecto
desde el comienzo del curso hasta la fase final que tuvo lugar el día
12 de mayo de 2011.\\

En esta edición se presentaron un total de 115 proyectos de diversa índole
y los resultados que obtuvo \wiki\ y \juego\ no pudieron ser más satisfactorios.
En la fase local del concurso fue galardonado con el premio al mejor
proyecto de \textit{Ocio} mientras que en la fase nacional, que se celebró en Granada,
se recibió el premio al mejor proyecto de \textit{Comunidad}.\\

El concurso se ha mostrado como un aliciente de lo más positivo para desarrollar
el proyecto con más entusiasmo y apertura hacia la comunidad. Ha sido muy
beneficioso en términos de audiencia y difusión gracias a los medios
que se han hecho eco de la convocatoria. Sin duda, ha sido uno de los
factores que más me ha motivado a seguir hacia delante tomándomelo como un
reto personal y buscando una experiencia enriquecedora junto al resto
de participantes.\\

\figura{vcusl.jpg}{scale=0.35}{Finalistas del V Concurso Universitario de Software Libre}{fig:vcusl}{h}

\section*{Creación de comunidad}

\wiki\ y \juego\ cuenta con un elevado aspecto de comunidad ya que juntos
forman una plataforma de aprendizaje de desarrollo de videojuegos en tres
dimensiones con \textsc{Ogre3D}. Era imprescindible llevar a cabo acciones
para difundir el proyecto y buscar la creación de dicha comunidad. La comunicación
con los lectores debía ser fluida, directa y cercana tratando de apelar
a su curiosidad e interés por la materia. A continuación, se listan los medios
empleados para contribuir a la difusión del proyecto.\\

\begin{itemize}
    \item \textbf{Blog de desarrollo}
    
    En mi blog personal se ha creado
    una sección especial para informar de los avances del proyecto y documentar
    los subsistemas y algoritmos más relevantes. En total se han redactado
    más de 70 artículos y se han recibido más de 85.000 visitas. Puede accederse
    desde la siguiente dirección.\\
    
    \url{http://siondream.com/blog/category/proyectos/pfc}\\
    
    \item \textbf{Forja}
    
    El proyecto se ha alojado en la forja de RedIRIS la cual no ha sido utilizada
    únicamente por su repositorio \textit{Subversion}. Se ha hecho uso
    de la sección de noticias para informar de los avances importantes,
    de la lista de tareas para gestionar el trabajo pendiente y de la lista
    de ficheros para publicar los resultados (juego y documentación adicional).
    El sitio del proyecto en la forja de RedIRIS puede ser accedido desde
    la siguiente dirección web.\\
    
    \url{https://forja.rediris.es/projects/cusl5-iberogre}\\
    
    \item \textbf{Twitter}
    
    Twitter es una red social de microblogging en la que los usuarios publican
    mensajes cortos. Es ampliamente utilizada para seguir noticias y estar
    informado de la actualidad en diversos sectores muy específicos. Existía
    una comunidad de desarrolladores hispanohablantes muy activa dentro de Twitter
    por lo que se decidió que el proyecto tuviera presencia en dicha red social.
    La comunicación en Twitter ha sido muy útil para acercarnos a los lectores
    y recibir sus sugerencias. Actualmente la cuenta $@$IberOgre posee 98
    seguidores y puede accederse desde:\\
    
    \url{http://twitter.com/#!/IberOgre}\\
    
    \item \textbf{Web en la forja}
    
    La forja de RedIRIS proporciona a los proyectos un pequeño espacio web.
    Si bien no otorga libertad absoluta para publicar contenido (solo
    se permiten webs estáticas en HTML) cuenta con un posicionamiento
    extremadamente favorable en buscadores. Se ha aprovechado dicho espacio
    con una web a modo de índice indicando brevemente en qué consiste el
    proyecto y enlazando a los medios oficiales. Puede verse en la figura
    \ref{fig:webproyecto} y accederse desde la siguiente dirección.\\
    
    \url{http://cusl5-iberogre.forja.rediris.es}\\
    
    \item \textbf{Canal de Youtube}
    
    Durante todo el desarrollo se han ido subiendo los progresos de \juego\
    al servicio de vídeo vía streaming por excelencia. Esto ha permitido
    que el interés por el proyecto crezca progresivamente. En total se han
    publicado diez vídeos los cuales se han reproducido en más de 2.800 ocasiones.
    Puede accederse al canal correspondiente en la siguiente dirección.\\
    
    \url{http://www.youtube.com/user/davidsaltares}\\
    
\end{itemize}

\figura{webproyecto.jpg}{scale=0.35}{Web del proyecto en la forja RedIRIS}{fig:webproyecto}{h}

\section*{Difusión}

Desde el momento en el que \wiki\ comenzó a contar con varios artículos
publicados, el público comenzó a tomar interés en el proyecto. Por supuesto,
el mayor empuje se produjo tras las fases local y final del V Concurso
Universitario de Software Libre. A continuación, haremos un repaso por los
medios que se han hecho eco del proyecto.\\

\begin{itemize}
    \item \textbf{Comunidades de desarrollo}
    
    Diversas comunidades de desarrollo de videojuegos en español mostraron
    rápidamente su interés principalmente en \wiki\ por proporcionar
    documentación en su idioma nativo. A continuación, enlazamos a algunos
    de los medios que han publicado artículos sobre el proyecto.\\
    
    \url{http://www.creagames.es/iberogre-un-proyecto-espanol-de-ogre-engine}\\
    \url{http://razonartificial.com/2011/03/iberogre-documentacion-de-ogre-en-espanol}\\
    \url{http://programandoideas.com/2011/01/iberogre-tutoriales-de-ogre3d-en-espanol}\\
    
    \item \textbf{Web oficial de Ogre3D}
    
    \wiki\ aparece mencionada en la cuarta edición de las noticias relacionadas
    con la comunidad de \textsc{Ogre3D}. La plataforma educativa fue visible
    en la portada de la web oficial del motor.\\
    
    \url{http://www.ogre3d.org/2011/03/01/ogre-news-4}\\
    
    \item \textbf{Twitter}
    
    El propio creador de \textsc{Ogre3D}, Steve Streeting, encontró la cuenta
    oficial de Twitter de \wiki\ y la recomendó públicamente. Desde entonces
    el número de seguidores y lectores aumentó considerablemente.\\
    
    \item \textbf{Prensa}
    
    Tras el V Concurso Universitario de Software Libre varios medios
    de prensa escrita publicaron una noticia al respecto. El proyecto
    apareció en Viva Cádiz, Diario de Cádiz, La Voz, la web oficial
    de la Universidad de Cádiz y varios medios más.\\
\end{itemize}
