\section{Metodología}

\wiki\ es un proyecto de documentación y, por tanto, no hemos podido
aplicar una metodología de desarrollo típica de sistemas software como el
RUP (\textit{Rational Unified Process}) empleado en \juego\ más adelante.
Podemos distinguir dos partes bien diferenciadas en este proyecto aunque
inseparables: el texto de los artículos y los ejemplos que los acompañan.\\

% Etapas para los artículos
Para la composición de todos los artículos se ha seguido un proceso
uniforme para asegurar la calidad de los mismos:\\

\begin{enumerate}
    \itemsep0em
    \item \textbf{Estudio de necesidades}: búsqueda de los puntos más
    relevantes a la hora de aprender a desarrollar videojuegos en tres dimensiones
    así como los apartados más fundamentales de la biblioteca \textsc{Ogre3D}.
    \item \textbf{Planificación}: estudio para encontrar la mejorar forma
    de abordar la materia escogida. División del contenido en una
    estructura uniforme con el resto de la plataforma de aprendizaje. 
    \item \textbf{Redacción y formateado}: recopilación de información y
    redacción del artículo completo. Adaptación del texto al formato
    de la plataforma, composición del artículo con imágenes que amenicen
    la lectura y publicación.
    \item \textbf{Desarrollo del ejemplo}: creación de un ejemplo descargable
    que ilustre todo el contenido expuesto en el texto del artículo de forma
    práctica. Creación de documentación adicional en el propio artículo
    para explicar el funcionamiento del ejemplo.
    \item \textbf{Evaluación}: tras la publicación del artículo se buscan
    las impresiones de los lectores y se anotan todas las sugerencias para
    mejorar el texto.
\end{enumerate}

\section{Análisis}
\label{iberogre-analisis}

En esta sección expondremos el análisis de la documentación sobre desarrollo
de videojuegos en tres dimensiones \wiki. En primer lugar realizaremos
una breve especificación de requisitos del sistema, posteriormente
hablaremos de los tipos de usuarios que accederán a la plataforma y finalmente
haremos un listado de los artículos necesarios.\\

\subsection{Especificación de requisitos del sistema}

\subsubsection{Requisitos de interfaces externas}

Los lectores accederán al contenido de \wiki\ mediante su monitor, empleando
el navegador de su elección. Por tanto, es importante que la plataforma emplee
estándares web soportados por todos los navegadores mayoritarios: Internet
Explorer, Mozilla Firefox, Google Chrome, Opera y Safari. Para conseguirlo,
será necesario escoger un sistema de publicación compatible con HTML 4.0
y CSS 2. En la medida de lo posible, deben seguirse las directrices del
\textit{World Wide Web Consortium} (W3C) \cite{website:w3c}.\\

El contenido deberá poder ser visualizado de forma clara en monitores
de varias resoluciones y relaciones de aspecto. El texto debe aparecer
de forma limpia, clara y poco recargada. No obstante, debe estar acompañado
de imágenes que amenicen la lectura de los temas más áridos. En caso de
que se adjunte código fuente, éste deberá poder visualizarse en la misma
plataforma empleando un resaltado de sintaxis apropiado.\\

La navegación debe ser intuitiva y no precisar de aclaraciones de ningún tipo.
El contenido debe estar perfectamente organizado en secciones temáticas
y ordenadas en dificultad creciente, de forma prácticamente secuencial.\\

\subsection{Requisitos sobre usuarios}

No todos los usuarios que accedan a \wiki\ tendrán los mismos privilegios.
Existirán tres categorías de usuarios cuyos privilegios se detallan a
continuación.\\

\begin{itemize}
    \itemsep0em
    \item \textbf{Lector}: usuario ocasional no registrado en la plataforma.
    \begin{itemize}
        \itemsep0em
        \item Acceso de lectura a los artículos publicados y a las secciones de la wiki.
        \item Posibilidad de enviar sugerencias de mejora.
    \end{itemize}
    \item \textbf{Usuario registrado}: 
    \begin{itemize}
        \itemsep0em
        \item Acceso de lectura a los artículos publicados y a las secciones de la wiki.
        \item Posibilidad de enviar sugerencias de mejora.
        \item Acceso de escritura a artículos existentes con el objetivo de
        ampliarlos, corregirlos y/o mejorarlos.
        \item Creación de nuevos artículos siguiendo la estructura general
        de la wiki.
    \end{itemize}
    \item \textbf{Administrador}: 
    \begin{itemize}
        \itemsep0em
        \item Acceso de lectura a los artículos publicados y a las secciones de la wiki.
        \item Posibilidad de enviar sugerencias de mejora.
        \item Acceso de escritura a artículos existentes con el objetivo de
        ampliarlos, corregirlos y/o mejorarlos.
        \item Creación de nuevos artículos siguiendo la estructura general
        de la wiki.
        \item Control sobre la estructura de la plataforma.
        \item Instalación de nuevas extensiones para mejorar la plataforma.
        \item Gestión del resto de usuarios.
    \end{itemize}
\end{itemize}

Cualquier persona que acceda al sitio web de la plataforma será considerado
como lector no registrado. En el momento en el que rellene el formulario
y envíe sus datos se convertirá en usuario registrado. Lo habitual es que
sólo exista un administrador del sistema o exista un grupo reducido de
administradores.\\

\subsection{Artículos}

% Bloques temáticos
Los artículos de \wiki\ están organizados en varios bloques temáticos
para abarcar todos los objetivos que nos hemos marcado en la introducción
de esta memoria. Esta organización también busca facilitar la navegación
al usuario de forma que acceda al contenido que desee en el menor tiempo
posible. Los bloques principales de \wiki\ son:\\

\begin{itemize}
    \itemsep0em
    \item \textbf{Primeros pasos}: bloque introductorio en el que el lector
    debe poder conocer la estructura, filosofía y metodología de la plataforma
    de aprendizaje. Tras leerlo, el usuario debe poder saber qué esperar
    de \wiki\ y cómo afrontar los contenidos posteriores.
    \item \textbf{Programación de videojuegos 3D}: sección que se dedicará
    a ofrecer los conocimientos matemáticos mínimos para desarrollar
    videojuegos en 3D. Se centrará en conceptos generales de geometría
    del espacio y álgebra lineal desde una perspectiva eminentemente práctica.
    Las matemáticas son un medio, una herramienta más para conseguir los objetivos
    propuestos. Al no ser el fin en sí mismo, priman las aplicaciones prácticas
    por encima de la rigurosidad (sin llegar a ser incorrectos).
    \item \textbf{Ogre3D}: bloque central que desglosa los apartados más
    relevantes del motor de renderizado \textsc{Ogre3D}. Tras terminar con
    este bloque, el lector debe ser capaz de utilizar la biblioteca
    para crear aplicaciones tridimensionales con elementos como modelos
    animados, sistemas de partículas y otros efectos de iluminación. Lo más
    importante es que se comprenda el funcionamiento del motor y se sea
    capaz de ampliar conocimientos de forma autónoma.
    \item \textbf{Otras tecnologías}: como ya se ha mencionado anteriormente,
    \textsc{Ogre3D} es sólo un motor de renderizado, no proporciona todos
    los elementos necesarios para construir un videojuego como gestión 
    de entrada del usuario, sonido, detección de colisiones o juego en red.
    Por ello, esta sección está enfocada a introducir en el uso de
    tecnologías complementarias a \textsc{Ogre3D} como \textsc{OIS} (entrada),
    \textsc{libSDL mixer} (audio) u \textsc{OgreBullet} (físicas).
    \item \textbf{Videojuegos}: en esta sección se adjuntará documentación
    sobre el desarrollo de videojuegos libres que hagan uso de \textsc{Ogre3D}.
    \juego\ contará con el primer artículo de esta sección. El videojuego pretende
    ser una aplicación práctica lo más realista posible del uso del motor y de la creación
    de videojuegos 3D en general. Adjuntaremos toda la documentación
    que se ha generado al respecto para que el lector pueda acceder a la misma
    de forma sencilla. Sin duda, es el bloque que se presta en mayor medida
    a ampliaciones futuras por parte de la comunidad.
\end{itemize}

\figura{bloques-iberogre.png}{scale=0.65}{Bloques temáticos de IberOgre}{fig:bloques-iberogre}{h}

% Lista de artículos

La sección \textit{Primeros pasos} únicamente cuenta con un artículo
introductorio.

\begin{itemize}
    \itemsep0em
    \item \textbf{Comenzando en IberOgre}: en este artículo a modo de preámbulo
    desgranamos todos los objetivos de la plataforma de aprendizaje. Posteriormente
    se explica la filosofía de trabajo que deseamos tomar con software compatible
    con varias plataformas y de carácter abierto.
\end{itemize}

En el bloque \textit{Programación de videojuegos 3D} se encuentran tres
artículos.

\begin{itemize}
    \itemsep0em
    \item \textbf{Introducción, puntos y vectores}: en el primer artículo
    sobre programación de videojuegos en 3D se hace una introducción al
    planteamiento que vamos a seguir durante todo el bloque. Posteriormente
    tratamos los puntos y los sistemas de coordenadas. Más adelante abarcamos
    los vectores, las operaciones que podemos realizar con ellos y sus aplicaciones
    en la materia. Finalmente se dedica una pequeña sección a hablar sobre
    interpolaciones lineales para suavizar movimientos.
    \item \textbf{Matrices}: en este artículo definimos el concepto de matriz
    desde el punto de vista matemático para después hacer un repaso por sus
    operaciones y finalmente tratar su utilidad a la hora de representar
    transformaciones en tres dimensiones (traslaciones, rotaciones
    y escalas).
    \item \textbf{Cuaternos}: los cuaternos son una extensión de los números
    reales que se suelen utilizar a la hora de representar rotaciones en un
    espacio tridimensional. En el artículo explicamos su definición, la forma
    de encadenar rotaciones y deshacerlas así como la operación necesaria
    para realizar interpolaciones y rotar un elemento de forma progresiva.
\end{itemize}

En el bloque principal, \textit{Ogre3D} tenemos los siguientes artículos.

\begin{itemize}
    \itemsep0em
    \item \textbf{Conociendo Ogre3D}: en el primer artículo relacionado
    con el motor de renderizado debe explicarse qué es capaz de hacer
    por nosotros \textsc{Ogre3D} y cuáles son sus limitaciones. Se expone
    una lista de características que incorpora así como los requisitos
    de hardware que exige para poder utilizarse. Comentamos su licencia
    permisiva y los lenguajes con los que es compatible (aunque nosotros
    sólo utilizaremos C++). Por último, hablamos brevemente de varias alternativas
    libres a \textsc{Ogre3D} como \textsc{Panda 3D} o \textsc{Irrlicht}.
    \item \textbf{Conceptos generales}: en este texto hacemos un recorrido
    por la arquitectura del motor, sus subsistemas más relevantes y la filosofía
    que se ha tomado a la hora de diseñarlos. Explicamos de forma rápida
    su objeto raíz, el gestor de recursos y el grafo de escena.
    \item \textbf{Instalación de Ogre3D 1.7 en GNU/Linux}: el lector
    debe poder acceder a una sencilla, rápida y directa guía de cómo instalar
    el entorno de desarrollo en su equipo para poder continuar con las lecciones
    de la plataforma de aprendizaje. En este artículo se expone el proceso
    de instalación para sistemas GNU/Linux. Se incluye un apartado
    especial para la popular distribución Ubuntu, ya que muchas de las dependencias
    están en los repositorios oficiales o en repositorios \textit{PPA}
    \cite{website:ppa}.
    \item \textbf{Instalación de Ogre3D 1.7 en Windows}: los usuarios de
    Windows también deben contar con una guía de instalación del entorno
    de desarrollo completo para poder seguir los contenidos de \wiki. En este
    caso es necesario instalar el compilador \textit{MinGW} y las bibliotecas
    \textsc{Ogre3D}, \textsc{Boost} y \textsc{DirectX}. Se precisa adjuntar un proyecto
    de prueba para comprobar la corrección del proceso.
    \item \textbf{Creación de un entorno de trabajo multiplataforma}: es
    altamente deseable que los usuarios de GNU/Linux y Windows trabajen
    de la misma forma siguiendo el contenido de la wiki. Para ello, los proyectos
    debían tener la misma estructura, el mismo código compatible y 
    un proceso de compilación similar. En este artículo se explica la
    jerarquía de directorios que tomaremos y los makefiles para compilar
    ejemplos y proyectos personales.
    \item \textbf{Inicialización y cierre de Ogre3D}: en este artículo
    creamos por primera vez una aplicación básica con \textsc{Ogre3D}. Se
    repasa la secuencia de inicialización y cierre del motor. Así mismo,
    conocemos las posibilidades de extensión del motor gracias al uso
    de plugins y complementos externos. Posteriormente aprendemos a configurar
    aspectos como la resolución, la frecuencia, la sincronización vertical
    utilizando ficheros de configuración. Finalmente se explica el sistema
    de logs de \textsc{Ogre3D} para depurar nuestras aplicaciones.
    \item \textbf{Gestión de recursos}: el motor incluye un sistema de gestión
    de recursos para optimizar el consumo de memoria evitando duplicidades
    y proporcionar un método uniforme de acceder a elementos como mallas
    tridimensionales, materiales o esqueletos. Los usuarios del motor
    deben conocer como gestionar el ciclo de vida de los recursos para poder
    cargar modelos entre otras muchas cosas.
    \item \textbf{Creación básica de escenas}: este artículo retoma la inicialización
    de \textsc{Ogre3D} y continua con el objetivo de ilustrar la creación
    de escenas. Debe mostrar como crear una ventana, un gestor de escenas,
    una cámara y un punto de vista. Una vez esté preparado el grafo de la escena
    para que se le añadan elementos, se hace un recorrido por la creación
    y gestión de nodos de escena con modelos tridimensionales estáticos.
    Se hace una breve introducción a la iluminación y se detallan las distintas
    aproximaciones para abordar el bucle de renderizado (bucle de juego).
    \item \textbf{Materiales}: nuestra biblioteca cuenta con un sistema
    de materiales que, básicamente, definen cómo un objeto interacciona
    con la luz que recibe. En \textsc{Ogre3D} existe una pequeña sintaxis
    para definir materiales. En este artículo hablamos de los detalles
    de los scripts para definir materiales y sus posibles parámetros así
    como de las maneras para cargar dichos materiales y aplicarlos a entidades.
    \item \textbf{Manipulación de nodos}: las escenas de \textsc{Ogre3D}
    están formadas por nodos que contienen mallas tridimensionales, 
    puntos de luz, cámaras u otros elementos. Para dotar de dinamismo
    a las escenas estos nodos deben trasladarse, ser rotados o escalados.
    En definitiva, deben ser dotados de movimiento. En este artículo
    hacemos un repaso por cómo se representan los vectores y transformaciones
    en el motor y comenzaremos a gestionar los nodos de la escena. Se
    exponen las diferencias entre los distintos espacios de transformación
    (local, parental y global) y aprendemos a aplicar transformaciones
    a los nodos (traslación, rotación y escalado). Finalmente se trata el tema
    de la interpolación entre puntos para producir movimiento suavizado.
    \item \textbf{Luces, sombras y entorno}: todos los videojuegos tridimensionales
    utilizan efectos de iluminación para crear la ambientación deseada.
    Era imprescindible incluir un artículo al respecto en \wiki. En la primera
    sección se habla sobre las distintas técnicas de sombreado disponibles
    en el motor. Posteriormente abarcamos el tema de la iluminación
    desde conceptos básicos como la reflexión difusa y especular hasta
    la creación de puntos que emitan luz en \textsc{Ogre3D}. El siguiente
    apartado corresponde al efecto de niebla, muy utilizado para ahorrar
    recursos o para crear una atmósfera terrorífica. Por último, se habla
    de los fondos en dos dimensiones para representar elementos como el cielo.
    \item \textbf{Animación}: la animación es un tema muy complejo pero imprescindible
    en todo videojuego. El artículo comienza con los conceptos básicos sobre
    la animación como el uso de fotogramas clave e interpolaciones entre los mismos. 
    Posteriormente explicamos los tres tipos de animación que existen en
    \textsc{Ogre3D} y cual es en la que nos centraremos durante el resto
    del texto. A continuación explicamos la técnica para cargar animaciones
    en el motor, reproducirlas y manipularlas. Por último, se hace una introducción
    a la mezcla de animaciones (\textit{animation blending}).
    \item \textbf{Sistemas de partículas}: en este artículo se abordan
    los sistemas de partículas de \textsc{Ogre3D} empleados para reproducir
    efectos como fuego, explosiones, humos, estelas mágicas, etc. Estos
    efectos se definen en scripts similares a los de los materiales y explicamos
    su estructura y sintaxis. Tras hacerlo comentamos el método para crear
    y destruir sistemas de partículas. Finalmente mencionamos un potente
    editor que nos ahorra tener que modificar de forma manual estos efectos.
    \item \textbf{Sistema de Overlays}: \textsc{Ogre3D} carece de sistema
    para crear elementos avanzados de interfaz como botones, etiquetas o
    formularios completos. En cambio, posee un sistema de overlays para mostrar
    elementos bidimensionales en una escena tridimensional. Estos paneles
    se definen también empleando scripts cuya estructura, sintaxis y propiedades
    detallamos a lo largo del artículo. Una vez terminadas las propiedades
    mostramos cómo cargar y gestionar overlays dentro del juego. Este sencillo
    sistema puede ser útil para usuarios que no requieran interfaces demasiado
    complejas.
\end{itemize}

Dentro de la sección \textit{Otras tecnologías} tenemos los siguientes
artículos.

\begin{itemize}
    \itemsep0em
    \item \textbf{Manejo básico de OIS}: es una biblioteca libre de gestión
    de dispositivos de entrada como joystick, ratón o teclado. Es la recomendada
    por \textsc{Ogre3D}, de hecho se distribuyen juntas. En este artículo
    aprendemos a consultar el estado de los dispositivos de entrada así como
    a responder a los eventos que estos producen en el momento adecuado empleando
    el patrón Observer \cite{gamm77}.
    \item \textbf{Exportar modelos desde Blender}: cuando creamos un modelo
    tridimensional posiblemente animado en una herramienta como \textit{Blender}
    debemos exportarlo al formato que emplea \textsc{Ogre3D}. En este artículo
    se detalla el proceso con todas las opciones que nos proporcionan los
    scripts de exportación.
    \item \textbf{Colisiones y físicas con OgreBullet}: como ya se ha mencionado,
    \textsc{Ogre3D} no proporciona un sistema de detección de colisiones ni
    simulaciones físicas. Por tanto, este artículo está destinado a documentar
    la instalación y el uso de la biblioteca \textsc{OgreBullet}, un envoltorio
    de la popular \textsc{Bullet}.
    \item \textbf{Extender la gestión de recursos, audio}: el sistema de gestión
    de recursos de \textsc{Ogre3D} permite ser extendido con sencillez para
    añadir nuevos tipos de recursos. Aprovechando esta capacidad explicamos
    el proceso incluyendo un sistema de audio utilizando la biblioteca
    \textsc{libSDL} en conjunción con \textsc{libSDL mixer}.
\end{itemize}



\section{Diseño}
\label{iberogre-diseno}

Una vez completado el análisis con los requisitos de \wiki\ nos centraremos
en el diseño de sus componentes. En primer lugar decidiremos la estructura
interna de cada uno de los artículos que compondrán la plataforma educativa
de desarrollo de videojuegos 3D. Posteriormente pasaremos al diseño de la
navegabilidad del sitio y de su apariencia visual. Finalmente especificaremos
los ejemplos con los que contarán los artículos.\\

\subsection{Estructura de los artículos}

La estructura lógica de los artículos queda reflejada en la figura \ref{fig:bloques-articulos}
y está pensada para que el lector adquiera de la manera más sencilla posible
los conocimientos necesarios para continuar. Dicha estructura ha ido siendo
refinada a lo largo del desarrollo gracias al continuo flujo de opciones
por parte de los lectores. La estructura consta de las siguientes partes.\\

\begin{itemize}
    \itemsep0em
    \item \textbf{Introducción}: breve resumen del campo que abarcará
    el artículo para que el lector sepa qué esperar cuando lo siga. Siempre
    se pretende ofrecer una perspectiva práctica dando ejemplos concretos
    de aplicaciones reales del contenido.
    \item \textbf{Requisitos previos}: breve sección con los artículos
    que el lector debería conocer antes de continuar. La estructura de 
    \wiki\ es aproximadamente secuencial pero en muchas ocasiones existen
    dependencias adicionales. Es importante que el lector sepa qué conocimientos
    debe poseer antes de enfrentarse a un texto sin entender nada.
    \item \textbf{Desarrollo}: bloque principal del artículo, que a su vez
    puede estar compuesto de secciones, en el que se desarrolla el tema a tratar.
    Se intercalan las explicaciones teóricas con pequeños fragmentos de
    código mostrando de forma práctica las técnicas comentadas. 
    \item \textbf{Ejemplo final}: en todos los artículos se adjunta un ejemplo
    final en forma de aplicación descargable que ilustra el contenido
    expuesto en el desarrollo. Al fichero descargable lo acompaña una explicación
    sobre cómo se ha implementado el ejemplo y se hacen sugerencias para
    que el lector modifique elementos. Este aspecto es muy relevante ya que
    la práctica es mucho más efectiva en el proceso de aprendizaje.
    \item \textbf{Conclusiones}: finalmente se incluye una sección que hace
    las veces de resumen de los contenidos tratados y lista las tareas
    que el lector debe ser capaz de realizar tras comprender el contenido
    del artículo.
\end{itemize}

\figura{bloques-articulos.png}{scale=0.65}{Estructura de los artículos en IberOgre}{fig:bloques-articulos}{h}

\subsection{Navegabilidad}

Para \wiki\ se ha elegido el motor \textit{MediaWiki} \cite{website:wikimedia} por
ser uno de los más empleados en este tipo de plataformas, contar con funcionalidades
más atractivas de personalización o edición y disponer de una comunidad
más numerosa dispuesta a ayudar en caso de problemas. Su diseño visual por
defecto ha sido modificado de forma que se adapte en la mayor medida posible
a la estructura de \wiki\ y a su enfoque.\\

En esta sección haremos un recorrido por el diseño de \wiki\ comenzando
por los bloques de su página principal. La pantalla de bienvenida prácticamente
al completo puede observarse en la figura \ref{fig:iberogre-diseno}, en
ella se aprecia claramente la división por bloques.\\

\figura{layout-iberogre.jpg}{scale=0.25}{Diseño de IberOgre}{fig:iberogre-diseno}{H}

\subsubsection{Bloque de bienvenida}

En el bloque de bienvenida se muestra el logo de \wiki\ y se hace una pequeña
introducción en pocas líneas sobre el tema que trata la plataforma. Es sencilla
y directa para captar el interés del lector potencial. Puede verse con mayor
detalle en la figura \ref{fig:iberogre-bienvenida}.\\

\figura{iberogre-bienvenido.jpg}{scale=0.25}{Bloque de bienvenida a IberOgre}{fig:iberogre-bienvenida}{h}

\subsubsection{Navegación}

El bloque de navegación se sitúa a la izquierda de la página y es útil para
acceder de forma directa a distintas secciones de interés para los usuarios
de la plataforma. Puede verse en la figura \ref{fig:iberogre-navegacion}
y se compone de los siguientes enlaces:

\begin{itemize}
    \itemsep0em
    \item \textbf{Página principal}: accede a la bienvenida de la wiki.
    \item \textbf{Portal de la comunidad}: página con información de interés
    para los lectores sobre el trabajo actual en \wiki.
    \item \textbf{Actualidad}: sección de noticias.
    \item \textbf{Cambios recientes}: permite ver cuáles son las páginas nuevas
    o las últimas ediciones, en qué han consistido y quién las ha realizado.
    \item \textbf{Página aleatoria}: nos lleva a cualquiera de los artículos
    de la wiki de forma aleatoria.
    \item \textbf{Ayuda}: breves consejos para aquellos interesados en colaborar
    con la plataforma.
\end{itemize}

\figura{iberogre-navegacion.jpg}{scale=0.5}{Panel de navegación en IberOgre}{fig:iberogre-navegacion}{h}

\subsubsection{Búsqueda}

El cuadro de búsqueda nos permite encontrar artículos que hablen sobre
las palabras claves introducidas. En primer lugar trata de buscar un artículo
cuyo nombre coincida exactamente con la clave introducida, posteriormente
busca coincidencias parciales en el título y, por último, dentro del texto.
Muestra los resultados por orden de relevancia de forma que el lector tenga
más probabilidades de encontrar lo que busca en la zona alta de la página
de resultados. El cuadro es sencillo y puede verse en la figura \ref{fig:iberogre-busqueda}.\\

\figura{iberogre-busqueda.jpg}{scale=0.55}{Cuadro de búsqueda en IberOgre}{fig:iberogre-busqueda}{h}

\subsubsection{Bloque de contenido}

El bloque más importante de \wiki\ es el de contenido, en él se listan todos
los artículos organizados por categorías y en orden creciente de dificultad.
A pesar de que las dependencias no son exactamente lineales en todos los casos,
es buena idea que el lector pueda empezar por el primero y continuar hacia
abajo por la lista. En la figura \ref{fig:iberogre-contenido} puede observarse
con más detalle este bloque aunque no se muestra en su totalidad por razones
de espacio.\\

\figura{iberogre-contenido.jpg}{scale=0.35}{Contenido de IberOgre}{fig:iberogre-contenido}{h}

\subsubsection{Bloque otros}

El bloque \textit{Otros} se divide en dos secciones diferenciadas. La primera
está relacionada con la difusión del proyecto y la comunicación con los usuarios
(\textit{IberOgre en otros medios}) mientras que la segunda contiene enlaces
a manuales de referencia con información para aquellos que deseen colaborar
o profundizar.

\begin{itemize}
    \itemsep0em
    \item \textbf{IberOgre en otros medios}: 
        \begin{itemize}
            \itemsep0em
            \item \textit{Blog oficial}: blog dedicado al desarrollo de \wiki\
            y \juego. Contiene tanto noticias sobre los avances como artículos
            documentando el desarrollo de subsistemas concretos.
            \item \textit{Forja}: enlace al repositorio de la forja de RedIRIS
            que aloja al proyecto. Contiene noticias, lista de tareas, lista
            de correo, ficheros descargables y el repositorio de código
            Subversion \cite{website:svn}.
            \item \textit{Twitter}: enlace a la cuenta oficial del proyecto
            en la conocida red social de microblogging. Como puede leerse
            en el capítulo \nameref{chap:comunidad}, se ha hecho un uso extensivo
            de esta herramienta durante todo el desarrollo. De esta manera,
            se establece un contacto más directo con los lectores.
            \item \textit{E-mail}: medio para contactar con la wiki y enviar
            sugerencias, críticas y otras opiniones para cambiar elementos
            e introducir mejoras.
        \end{itemize}
    \item \textbf{Ayuda}: 
    \begin{itemize}
            \itemsep0em
            \item \textit{Manual oficial de Ogre3D en español}: traducción completa
            del manual oficial de referencia de \textsc{Ogre3D} en español.
            El fichero \textit{PDF} ha sido ofrecido por el colaborador
            Mario Velázquez Muñoz. Es un manual de consulta que complementa
            perfectamente al carácter didáctico de los artículos.
            \item \textit{Edición en MediaWiki}: guía sencilla sobre edición
            en sistemas \textit{MediaWiki} creada por Noelia Sales Montes
            y Emilio José Rodríguez Posada. Aquellos lectores interesados
            en colaborar pueden hacerlo aunque no tengan experiencia con plataformas
            de este tipo utilizando este sencillo manual.
        \end{itemize}
\end{itemize}

\figura{iberogre-otros.jpg}{scale=0.35}{Otros temas relacionados con IberOgre}{fig:iberogre-otros}{h}

\subsubsection{Artículos}

Cada artículo comienza con una introducción breve como ya hemos mencionado
anteriormente. A esta introducción le sigue un índice de contenidos con
enlaces que llevan directamente a las subsecciones del texto. Esta
funcionalidad
facilita el acceso a aquellos lectores que deseen buscar un apartado concreto.\\

\subsection{Ejemplos}

Los ejemplos en \wiki\ son una parte imprescindible del aprendizaje, proporcionan
una aplicación práctica del contenido teórico del texto que los preceden.
Estos ejemplos tratan de aglutinar lo que debería aprenderse tras leer el
artículo en una sola y sencilla aplicación descargable. Recordemos que en este
caso es más relevante la sencillez que la vistosidad y la calidad del código.
Tras cada ejemplo, se anima al lector a estudiar el código y realizar
modificaciones para demostrar su comprensión del mismo y dominio de la materia.
A continuación hacemos una breve descripción sobre cada uno de los ejemplos
de \wiki.\\

\subsubsection{Inicialización y cierre de Ogre}

Tras el artículo \textit{Inicialización y cierre de Ogre} el lector no sabe
mostrar nada en pantalla, ni siquiera crear una ventana en la que renderizar
elementos. Simplemente conoce la secuencia de inicialización del motor y
la forma para salir de una aplicación de forma ordenada y correcta. El ejemplo
se limita a crear el objeto principal de \textsc{Ogre3D} habiendo configurado
previamente el sistema de logs y habiendo mostrado un diálogo de configuración
al usuario. Tal y como se muestra en la figura \ref{fig:ejemplo-inicializacion},
simplemente inicializamos el motor para cerrarlo seguidamente.\\

\figura{ejemplo-inicializacion.jpg}{scale=0.5}{Ejemplo de inicialización y cierre de Ogre3D}{fig:ejemplo-inicializacion}{h}

\subsubsection{Creación básica de escenas}

En este artículo aprendíamos a crear una ventana de renderizado, configurar
el gestor de escenas, utilizar la gestión de recursos, cargar elementos
dentro del grafo de la escena y controlar el bucle de renderizado empleando
varias aproximaciones. El ejemplo aglutina todos estos conceptos creando
una sencilla aplicación que inicializa \textsc{Ogre3D}, carga un personaje
en pantalla y permanece a la espera de algún evento de salida. El resultado
puede verse claramente en la figura \ref{fig:ejemplo-escenas}. Se requieren
conocimientos de gestión de eventos de entrada con \textsc{OIS}, por lo que
es recomendable leer dicho artículo y seguir su ejemplo en primer lugar.\\

\figura{ejemplo-escenas.jpg}{scale=0.35}{Ejemplo de creación de escenas}{fig:ejemplo-escenas}{h}

\subsubsection{Materiales}

En el artículo sobre materiales en \textsc{Ogre3D} aprendemos a escribir
scripts definiendo materiales, a cargarlos en nuestra aplicación y a aplicarlos
a entidades como mallas tridimensionales. En el ejemplo para este artículo
tenemos una sencilla escena con una paredes y una esfera en el centro iluminada
por varios puntos de luz. Empleando el teclado numérico podemos cambiar el material
de la esfera para ver los efectos producidos. De esta manera se resume la
creación de materiales mediante scripts, su carga en memoria y aplicación
sobre entidades, puede verse una captura del ejemplo en ejecución en la 
figura \ref{fig:ejemplo-materiales}.

\figura{ejemplo-materiales.jpg}{scale=0.4}{Ejemplo de materiales}{fig:ejemplo-materiales}{h}

\subsubsection{Manipulación de nodos}

En el artículo sobre manipulación de nodos aprendemos a gestionar la información
básica de los nodos como visibilidad, nombre, sus descendientes, etc.
Partimos de la escena con el personaje vista en el
ejemplo de la creación básica de escenas. Utilizando la gestión de eventos
proporcionada por \textsc{OIS} capturamos las pulsaciones de las teclas
W, A, S y D para desplazar y rotar al personaje por el mundo. Empleando
la rueda del ratón podemos escalarlo. Puede verse una captura en la figura
\ref{fig:ejemplo-nodos}.\\

\figura{ejemplo-nodos.jpg}{scale=0.6}{Ejemplo de manipulación de nodos}{fig:ejemplo-nodos}{h}

\subsubsection{Luces, sombras y entorno}

En el artículo sobre iluminación aprendemos a gestionar varios efectos
lumínicos y de ambientación como la niebla, las fuentes de luz, las técnicas
de sombreado y los fondos. El ejemplo trata de aglutinar todos estos conceptos
en una aplicación interactiva que nos permite activar y desactivar varios
de los efectos anteriormente mencionados. Se crea una escena con un
plano texturizado a modo de suelo y un personaje sobre él. Los controles
son los siguientes y el resultado puede observarse en la figura \ref{fig:ejemplo-iliuminacion}.

\begin{itemize}
    \itemsep0em
    \item \textbf{s}: alterna entre distintas técnicas de sombreado. Es
    la mejor forma de ver la diferencia entre ambas.
    \item \textbf{d}: activa o desactiva el fondo.
    \item \textbf{n}: activa o desactiva la niebla.
    \item \textbf{1}: apaga o enciende la luz número 1 (roja).
    \item \textbf{2}: apaga o enciende la luz número 2 (azul).
    \item \textbf{3}: apaga o enciende la luz número 3 (verde).
    \item \textbf{4}: apaga o enciende la luz número 4 (amarilla).
\end{itemize}

\figura{ejemplo-iluminacion.jpg}{scale=0.35}{Ejemplo de iluminación}{fig:ejemplo-iliuminacion}{h}

\subsubsection{Animación}

En el artículo de animación se exponen las distintas técnicas para animar
entidades en \textsc{Ogre3D} así como la forma de activar y mezclar
varias animaciones al mismo tiempo. En el ejemplo cargamos a la mascota del
motor llamada \textit{Simbad} sobre un sencillo escenario y ofrecemos
controles al usuario para animarlo. Utilizando las teclas de dirección
\textit{Simbad} se desplaza con una animación y pulsando \textit{D} comienza
o para de bailar. Puede verse el resultado en la figura \ref{fig:ejemplo-animacion}.\\

\figura{ejemplo-animacion.jpg}{scale=0.55}{Ejemplo de animaciones}{fig:ejemplo-animacion}{h}

\subsubsection{Sistemas de partículas}

En este artículo se explican las técnicas de definición de sistemas de partículas
a través de scripts en texto plano así como su carga dentro de una escena.
En el ejemplo simplemente cargamos varios sistemas de partículas entre
los que el usuario puede alternar empleando las teclas numéricas del 1 al
9. El resultado puede verse en la figura \ref{fig:ejemplo-particulas}.\\

\figura{ejemplo-particulas.jpg}{scale=0.30}{Ejemplo de sistemas de partículas}{fig:ejemplo-particulas}{h}

\subsubsection{Sistema de Overlays}

La lección sobre el sistema de Overlays de \textsc{Ogre3D} incluye la sintaxis
y propiedades de los scripts que los definen así como su carga dentro del juego.
En el ejemplo se incluyen varios scripts para definir estas plantillas bidimensionales
ofreciendo cierta interactividad. Partimos del resultado del ejemplo de materiales
y añadimos un panel informativo con el nombre del material seleccionado
y el número de cuadros por segundo (\textit{FPS}) a la que se ejecuta
la aplicación. Si cambiamos de material con el teclado numérico, veremos
el cambio reflejado en el panel como se aprecia en la figura \ref{fig:ejemplo-overlays}.\\

\figura{ejemplo-overlays.jpg}{scale=0.25}{Ejemplo de Overlays}{fig:ejemplo-overlays}{h}

\subsubsection{Manejo básico de OIS}

\textsc{OIS} es la biblioteca que utilizamos en \wiki\ para la gestión
de dispositivos de entrada de usuario. En el artículo se detalla el proceso
de inicialización, configuración y cierre de la biblioteca así como el manejo
de dispositivos (joysticks, ratones y teclados) para consultar su estado
en un momento dado. También se explica la aproximación empleando \textit{Listeners}
\cite{gamm77} para dar respuesta a eventos. En el ejemplo se crea una aplicación
de \textsc{Ogre3D} sin ventana y permanecemos a la espera de eventos que
mostramos por la terminal.\\

\subsubsection{Extender la gestión de recursos, audio}

En este artículo se explica el proceso de extensión del sistema de gestión
de recursos de \textsc{Ogre3D} a través del desarrollo de un subsistema
de audio empleando la biblioteca \textsc{libSDL mixer}. En el ejemplo
se carga una escena con un escenario, una silla y un micrófono a modo
de teatro para monólogos. El usuario se pone en el papel de regidor
indicando qué sonido o melodía debe reproducirse en un momento dado empleando
el teclado numérico, se muestra un sencillo panel con la leyenda.

\begin{itemize}
    \itemsep0em
    \item \textbf{1}: sintonía de comienzo del programa.
    \item \textbf{2}: sintonía del contenido del programa.
    \item \textbf{3}: sintonía del final del programa.
    \item \textbf{4}: aplausos de un público ficticio.
    \item \textbf{5}: risas enlatadas.
    \item \textbf{6}: abucheos del público ficticio.
\end{itemize}

El resultado puede verse en la figura \ref{fig:ejemplo-audio}.\\

\figura{ejemplo-audio.jpg}{scale=0.3}{Ejemplo de extensión del sistema de recursos, audio}{fig:ejemplo-audio}{h}


\section{Implementación}

Tras enumerar los usuarios, artículos, detallar la estructura de los mismos
y la navegabilidad de la plataforma pasamos a la fase de implementación.
Nos centraremos en los puntos más reseñables de esta fase. En primer lugar
hablaremos de la elección del motor de wikis \textit{MediaWiki} y de sus
características principales. Posteriormente trataremos las plantillas que
hemos empleado para formatear la documentación. Finalmente abordaremos
la estructura general de los ejemplos. La escritura de los artículos 
en sí mismos es trivial y por ello no es comentada en esta sección.\\

% MediaWiki
\subsection{El motor MediaWiki}

\textit{MediaWiki} es un popular motor para aplicaciones wikis desarrollado
por la propia Fundación Wikimedia y utilizado en todos sus sitios como
Wikipedia, Wiktionary y Wikinews entre otros muchos. Está escrito en el
lenguaje de programación PHP y precisa de una base de datos de tipo SQL
para almacenar la información. Su primera versión fue liberada en enero
de 2002 y actualmente, la 1.17.0 es la última versión estable y fue lanzada
en junio de 2011. \cite{website:mediawiki}.\\

\figura{mediawiki.png}{scale=0.3}{Logo del motor MediaWiki}{fig:mediawiki}{h}

\textit{MediaWiki} es, con diferencia, el software para creación y gestión
de wikis más utilizado en todo el mundo. Es fácilmente extensible mediante
complementos y existen bots que automatizan tareas como la prevención
y corrección de vandalismos. Uno de estos bots es AVBot desarrollado
por Emilio José Rodríguez Posada \cite{website:avbot}.\\

Escribir en la sintaxis utilizada por el motor es bastante sencillo, ya que
utiliza un lenguaje mucho más simple que HTML. Esto es muy importante ya
que buscamos colaboradores para la plataforma educativa. Cuanto más accesible
sea, más probabilidades de éxito. Por ello se ha adjuntado la guía
redactada por Noelia Sales Montes y Emilio Rodríguez Posada \textit{Edición
de wikis WikiMedia} \cite{pdf:wikimedia}.\\

El sistema proporciona un método de edición sencillo y una navegación
cómoda con contenido multimedia. Podemos crear espacios de nombres y
secciones bien diferenciadas. Es altamente configurable y podemos ajustarlo
para que encaje perfectamente con los objetivos de \wiki. Del proceso de
instalación y configuración se ha encargado la Oficina de Software Libre
de la Universidad de Cádiz ya que dicha instalación se ha realizado en
sus servidores.\\

Su extensibilidad, posibilidades de configuración, la cantidad de documentación
disponible y la enorme y activa comunidad han hecho que nos decantemos
por el uso de \textit{MediaWiki} como software para dar soporte a \wiki.\\

% Plantillas
\subsection{Plantillas}

Las plantillas son fragmentos de código que aceptan parámetros a los que
podemos llamar desde otras páginas utilizando una sintaxis especial. Esto nos
ayuda a generalizar ciertas estructuras y ahorrar en recursos por parte del
servidor. En esta sección haremos un recorrido por todas las plantillas utilizadas,
sus parámetros y el resultado que producen al utilizarlas.\\

\begin{itemize}
    \item \textbf{Artículo}
    
    Esta plantilla se utiliza por cada artículo en el bloque de contenido.
    Incluye el nombre con el enlace al texto, una descripción y un icono
    representativo. Su código es el siguiente, pueden verse los parámetros
    identificados por su orden de aparición en la llamada a la plantilla
    y encerrados en llaves triples.
    
    \begin{lstlisting}[style=wiki]
{| style= "border: 0;
           margin: 0;
           background-color: inherit;"
           cellpadding="3"
           
 | valign="top" | [[Imagen:{{{1}}}|64px|link={{{2}}}]]
 | valign="top" | 
 | '''[[{{{2}}}]]'''<br>{{{3}}}
|}
    \end{lstlisting}
    
    Una llamada como esta:

    \begin{lstlisting}[style=wiki]
{{ Articulo
|fuego.png
|Sistemas de particulas
|Aprende a crear efectos especiales con Ogre como llamas, explosiones
 o nubes de humo para mejorar tus videojuegos.
}}
    \end{lstlisting}

    Produce el resultado de la figura \ref{fig:articulo}:\\
    
    \figura{articulo.jpg}{scale=0.4}{Plantilla Artículo}{fig:articulo}{h}

    
    \item \textbf{RecursoExterno}
    
    Para referenciar enlaces externos de una forma similar a los artículos
    se emplea esta plantilla. Cuenta con un título que lleva al destino
    y un icono relacionado que hace lo propio. Además, se añade una descripción.
    El código que consigue el efecto es el siguiente.
    
    \begin{lstlisting}[style=wiki]
{| style= "border: 0;
           margin: 0;
           background-color: inherit;"
           cellpadding="3"
           
| valign="top" | [[Imagen:{{{1}}}|64px|link={{{2}}}]]
| valign="top" | 
| '''[{{{2}}} {{{3}}}]'''<br>{{{4}}}
|}
    \end{lstlisting}
    
    Una llamada como la siguiente:
    
\begin{lstlisting}[style=wiki]
{{RecursoExterno
|cuaderno.png
|http://siondream.com/blog/category/proyectos/pfc/
|Blog oficial
|Noticias, experiencias y avances en IberOgre
}}
    \end{lstlisting}
    
    Produce el resultado que puede apreciarse en la figura \ref{fig:recursoexterno}.\\
    
    \figura{recursoexterno.jpg}{scale=0.45}{Plantilla RecursoExterno}{fig:recursoexterno}{h}
    
    \item \textbf{BloqueSeccion}\\
    
    Esta plantilla se emplea para conseguir el bloque de que preceden
    a las secciones de la wiki dentro del bloque contenido. Su código
    es el siguiente.
    
    \begin{lstlisting}[style=wiki]
<div style="padding-top:2px;
            padding-bottom:2px;
            padding-left:5px;
            padding-right:2px;
            border:1px solid #092C00;
            margin: 4px;
            background-color: #F0FFF0;">
'''{{{1}}}'''
</div>
    \end{lstlisting}
    
    \item \textbf{Calendario}
    
    En diversas páginas interesaba colocar un calendario con la fecha
    de la edición. Por ello, se ha creado esta plantilla que coloca la fecha
    actual en un formato de calendario clásico. Su código wiki 
    y HTML es el siguiente.
    
    \begin{lstlisting}[style=wiki]
<div style= "border:solid #ccc;
             background: #fff;
             border-width: 1px 3px 3px 1px;
             text-align: center;
             padding-top:3px;
             float:left;
             font-size: smaller;
             line-height: 1.3;
             margin-right: 4px;
             width: 7em">
             
    [[{{CURRENTDAYNAME}}]]
    [[{{CURRENTDAY}} de {{CURRENTMONTHNAME}}|
    <span style= "font-size: x-large;
                  width: 100%;
                  display: block;
                  padding:6px 0px">
        {{CURRENTDAY}}
    </span>]]
    
    <span style="display: block;"> [[{{CURRENTMONTHNAME}}]]</span>
    
    <span style= "background: #aaa;
                  color: #000;
                  display: block;">
    '''[[{{CURRENTYEAR}}]]'''
    </span>
</div>
    \end{lstlisting}
    
    El resultado producido es el que podemos ver en la figura \ref{fig:calendario}.\\
    
    \figura{calendario.jpg}{scale=0.45}{Plantilla Calendario}{fig:calendario}{h}
    
    \item \textbf{FicheroDescargable}
    
    Los ejemplos finales de cada artículo de \wiki\ pueden descargarse
    en forma de paquete para ser compilados, ejecutados o modificados
    por los lectores. Para ello, empleamos una plantilla que añade
    un enlace al fichero descargable, el título del artículo una
    descripción del ejemplo. Su código es el siguiente:
    
    \begin{lstlisting}[style=wiki]
{| style= "border: 0;
           margin: 0;
           background-color: inherit;"
           cellpadding="3"
| valign="top" | [[Imagen:{{{1}}}|64px|]]
| valign="top" | 
| '''[[media:{{{2}}}|{{{3}}}]]'''<br>{{{4}}}
|}
    \end{lstlisting}
    
    Empleado la siguiente llamada:
    
        \begin{lstlisting}[style=wiki]
{{FicheroDescargable
|ejemplo.png
|ogre_escenas.zip
|Ejemplo de creacion basica de escenas
|Pequena aplicacion que inicia Ogre y carga un personaje en la escena
}} 
    \end{lstlisting}
    
    Obtenemos el siguiente bloque (ver figura \ref{fig:ficherodescargable}).\\
    
    \figura{ficherodescargable.jpg}{scale=0.55}{Plantilla FicheroDescargable}{fig:ficherodescargable}{h}
    
\end{itemize}

% Estructura general de los ejemplos
\subsection{Estructura general de los ejemplos}

Todos los ejemplos de \wiki\ se han implementado de manera similar aunque
cada uno tiene sus particularidades en los aspectos concretos. Esta decisión
viene motivada por la intención de conseguir uniformidad tanto en el estilo
como en las soluciones. De esta forma, los lectores comprenderán los ejemplos
cada vez con mayor facilidad y se centrarán únicamente en las partes cambiantes,
lo realmente interesante y lo que se imparte en cada lección.\\

El esquema general puede verse en el diagrama de clases de diseño en la
figura \ref{fig:ejemplo-clases}. La clase \textit{AplicacionOgre} es la encargada
de iniciar el motor \textsc{Ogre3D} y la biblioteca \textsc{OIS} a través
de los métodos privados \textit{iniciarOgre()} e \textit{iniciarOIS()}.
Hereda de las clases observadoras \cite{gamm77} \textit{WindowEventListener},
\textit{KeyListener} y \textit{MouseListener}. La primera se encarga de
capturar eventos de pantalla (movimiento, cierre y redimensión), la segunda
toma eventos de teclado (pulsar y soltar tecla) mientras que la tercera
hace lo propio con el ratón (movimiento, pulsar botón y liberar botón).
Como puede verse, existe un gestor de evento para cada uno de ellos.\\

\figura{ejemplo-clases.jpg}{scale=0.45}{Diagrama de clases para los ejemplos}{fig:ejemplo-clases}{h}

Con una llamada a \textit{buclePrincipal()} comienza el bucle de juego
hasta que se presione la tecla escape o se cierre la ventana. Esta clase
está preparada para proporcionar lo básico de una aplicación interactiva
que use \textsc{Ogre3D}. Está pensada para que heredemos de ella en cada
ejemplo que creemos. Por este motivo, el bucle de juego y los gestores
de eventos son virtuales, podemos sobrecargar su comportamiento.\\

Efectivamente, en cada ejemplo creamos una clase hija que suele llevar
por nombre \textit{EscenaSimple}. Esta clase se encarga de crear, gestionar
la escena que se desarrolle en el ejemplo y controlar su lógica (interactividad).
Siempre cuenta con varios métodos de configuración: \textit{prepararRecursos()},
\textit{configurarSceneManager()}, \textit{crearCamara()} y \textit{crearEscena()}.
En cada ejemplo concreto se pueden añadir más métodos y atributos según
sea necesario.\\

Esta aproximación centrada en la reutilización de código y la sencillez
ha facilitado la implementación de ejemplos de forma rápida y directa. Además,
como ya hemos mencionado, facilita el aprendizaje a los lectores. Para
saber más sobre los ejemplos concretos, puede accederse a cualquier artículo
de \wiki\ y descargar el paquete correspondiente.\\

\section{Pruebas}

Un plan de pruebas elaborado nos garantiza un mínimo de calidad para
el proyecto. En este caso no podemos aplicar el concepto de prueba
de caja blanca ni prueba de caja negra ya que hablamos de un proyecto
de documentación (exceptuando los ejemplos finales de cada artículo).
En esta sección enumeraremos las pruebas que se han realizado sobre la 
plataforma de aprendizaje en su totalidad.\\

Es necesario mencionar que no se han realizado pruebas en un servidor de
preproducción ya que la instalación y configuración corría a cargo de la
Oficina de Software Libre de la Universidad de Cádiz. No obstante, el motor
\textit{MediaWiki} ha sido probado de forma extensiva y es muy estable
a día de hoy.\\

La plataforma era pública (aunque no abierta a ediciones) para todo el mundo
y con los primeros artículos llegaron los primeros lectores. \wiki\ ha sido
sometido a pruebas por parte de muchos usuarios. Éstos han podido ofrecer
su opinión a través del blog de desarrollo, de Twitter o del correo
electrónico.\\

A continuación adjuntamos la lista de pruebas realizadas junto a los
resultados obtenidos.\\

\begin{enumerate}
    \itemsep0em
    \item \textbf{Estructura de los artículos}
    
    \textit{¿Se cumple con la estructura acordada? ¿Cumple la estructura
    con los objetivos?}\\
    
    \wiki\ se ha dividido en las cuatro secciones establecidas durante la
    fase de análisis (capítulo \ref{iberogre-analisis} y página \pageref{iberogre-analisis}).
    Los lectores han manifestado que la división en introducción, matemáticas
    \textsc{Ogre3D} y otras tecnologías les facilitó la organización
    de su aprendizaje. Por tanto, se han cumplido con los objetivos y el
    resultado de la prueba es positivo.\\
    
    \item \textbf{Errores en el texto}
    
    \textit{¿Existen errores ortográficos, de sintaxis o expresión en los
    textos?}\\
    
    Cada artículo ha sido revisado tanto por su redactor inicial como por
    varios usuarios que se han prestado a colaborar con las correcciones.
    El uso del corrector ortográfico se ha combinado con lecturas humanas
    para detectar y eliminar los fallos de este tipo.\\
    
    \item \textbf{Plantillas}
    
    \textit{¿Las plantillas se visualizan correctamente? ¿Existen plantillas
    que devuelvan algún error?}\\
    
    Se ha comprobado que todas las llamadas a plantillas funcionan como
    se esperaba de ellas. Todas las llamadas se visualizan de forma correcta.\\
    
    \item \textbf{Etiquetado}
    
    \textit{¿Están todos los artículos debidamente etiquetados?}\\
    
    En \textit{MediaWiki} los artículos pueden etiquetarse de forma que
    encontrarlos sea más sencillo. En definitiva facilita la navegación
    ya que se crea una página de etiquetas listando a los artículos
    que contienen. En \wiki\ todos los artículos han sido etiquetados
    por lo que el resultado del test es positivo.\\
    
    \item \textbf{Usuario no registrado}
    
    \textit{¿Son los permisos de estos usuarios los correctos? ¿Pueden
    editar artículos o crear otros nuevos?}\\
    
    Como se dice en el análisis de la wiki, los usuarios no registrados
    pueden acceder a todo el contenido de \wiki\ aunque no pueden hacer
    ningún tipo de modificación. Se ha comprobado cerrando la sesión de
    nuestro usuario que, sin estar registrados en el sistema, no podemos
    crear ni modificar página alguna. Por tanto, el resultado es positivo.\\
    
    \item \textbf{Usuario registrado}
    
    \textit{¿Pueden los usuarios registrados crear o modificar contenido?
    ¿Tienen acceso de administración?}\\
    
    Tal y como se menciona en la prueba anterior, ya habíamos definido
    los permisos que debía tener cada usuario en el análisis. El usuario
    registrado debía poder acceder tanto al contenido en modo lectura
    como en modo escritura. Asimismo debía contar con la capacidad
    para crear nuevos artículos. No obstante, no se le permite el acceso
    a la configuración del sistema.\\
    
    Se han probado que puede trabajar con el contenido pero no así con la
    configuración de \wiki. Sólo el usuario administrador cuenta con
    acceso al servidor de producción y a la base de datos.\\
    
    \item \textbf{Usuario administrador}
    
    \textit{¿Cuenta el usuario administrador con todos los permisos?}\\
    
    El usuario administrador cuenta con los privilegios del usuario registrado
    y, además, debe poder acceder a parámetros de configuración así como
    poder instalar nuevas extensiones o gestionar las actuales. Durante el
    desarrollo, fueron necesarias varias tareas administrativas y este usuario
    pudo resolverlas sin problemas confirmando que el esquema de permisos
    funciona como se había planeado.\\
    
    \item \textbf{Ayuda}
    
    \textit{¿Es la ayuda lo suficientemente clara para que los lectores
    puedan empezar a colaborar? ¿Es ampliable la plataforma? ¿Los usuarios
    pueden enviar opiniones?}\\
    
    La Ayuda en \wiki\ es de extrema importancia ya que debe informar a los
    usuarios de las posibilidades que tienen para colaborar y animarles
    a hacerlo. La guía de edición en \textit{WikiMedia} de Noelia Sales Montes
    y Emilio José Rodríguez Posada cumplió su cometido ya que, lectores
    que nunca habían editado una wiki con este motor pudieron hacerlo.\\
    
    La plataforma se ha mostrado ampliable en gran medida. Tras la conclusión
    de este proyecto, la wiki cuenta con temas en los que se puede profundizar
    aún más. Algunos usuarios ya se han mostrado interesados en colaborar.\\
    
    En la página inicial de \wiki\ se adjuntan enlaces tanto al blog oficial
    de desarrollo, como a la cuenta oficial de Twitter como al correo electrónico.
    Los usuarios pueden ponerse en contacto (y de hecho lo han hecho en numerosas
    ocasiones) a través de cualquiera de estos medios.\\
    
    \item \textbf{Ejemplos}
    
    \textit{¿Funcionan correctamente los ejemplos? ¿Abarcan el contenido
    de los artículos? ¿Están debidamente documentados?}\\
    
    Esta prueba se ha aplicado a todos los ejemplos de \wiki\ y en la totalidad
    de los casos el resultado ha sido satisfactorio. Lo especificado en la
    fase de diseño de los ejemplos (sección \ref{iberogre-diseno}, página
    \pageref{iberogre-diseno}) se cumple con exactitud.\\
    
    Su diseño tiene como objetivo ilustrar todo el contenido del texto que
    precede a cada ejemplo y siempre se logra cumplir todo el temario. Cada
    ejemplo viene acompañado de una explicación que informa de su cometido,
    su diseño y funcionamiento. Esto, junto a las críticas positivas por
    parte de los usuarios, nos hace considerar que están debidamente
    documentados.\\
    
    \item \textbf{Multiplataforma}
    
    \textit{¿Es todo el código expuesto en la wiki multiplataforma?
    ¿Funcionan todos los ejemplos tanto en Windows como en GNU/Linux?}\\
    
    Uno de los objetivos principales de \wiki\ era ofrecer directrices
    sobre desarrollo de software multiplataforma, al menos para GNU/Linux
    y Windows. Todo el código de la plataforma de aprendizaje ha sido
    probado en ambos sistemas operativos con éxito.\\
    
    Cada ejemplo ha sido compilado y ejecutado en Ubuntu (GNU/Linux) y
    en Windows 7 utilizando el método expuesto en el artículo
    de creación de un entorno de desarrollo multiplataforma \cite{website:multiplataforma}
    obteniendo resultados positivos.\\
    
\end{enumerate}
