En este capítulo se expone la planificación de tareas que se ha seguido
para desarrollar \wiki\ y \juego. En primer lugar se adjunta el diagrama
de Gantt completo para, más adelante, complementarlo con un breve comentario de
cada tarea.

\section{Diagrama de Gantt}

Como puede observarse en el diagrama de Gantt, en tareas relacionadas con
el apartado artístico de \juego\ participan más personas. Esto se debe a que
no poseía ni poseo los conocimientos o destrezas requeridos para crear
modelos tridimensionales animados, piezas musicales ni grabar y procesar
efectos de sonido. Por ello, he contactado con expertos en dichas materias
dispuestos a colaborar en el desarrollo de un videojuego completamente libre.
Finalmente, los participantes adicionales son:

\begin{itemize}
    \itemsep0em
    \item Antonio Jiménez Rodríguez: artista 3D encargado de diseñar, modelar,
    texturizar al protagonista y a los enemigos del juego.
    \item Daniel Belohlavek: artista 2D encargado de los iconos de los hechizos.
    \item Daniel Pellicer García y Antonio Caro Oca: encargados de componer
    y producir la banda sonora completa del juego.
    \item Francisco Martín Márquez: técnico de sonido encargado de grabar
    y procesar los efectos de sonido para el juego.
\end{itemize}

Por cuestiones de espacio y formato, el diagrama de Gantt ha sido divido
en dos. En la figura \ref{fig:planificacion1} puede observarse la planificación
del proyecto correspondiente al intervalo comprendido entre julio de 2010
y enero de 2011. Por otro lado, la figura \ref{fig:planificacion2} muestra
el resto de la planificación, desde febrero de 2011 hasta septiembre de 2011.

\figura{planificacion1.jpg}{scale=0.35}{Planificación del proyecto desde julio de 2010 hasta enero de 2011}{fig:planificacion1}{H}


\figura{planificacion2.jpg}{scale=0.35}{Planificación del proyecto desde febrero de 2011 hasta septiembre de 2011}{fig:planificacion2}{H}


\section{Etapas de desarrollo del proyecto}

\begin{enumerate}
    \itemsep0em
    \item \textbf{Planificación}
    
    Dada la envergadura del proyecto, era necesaria
    una etapa de planificación en la que se ha estudiado de forma cuidadosa
    el alcance del mismo y las posibles dificultades a encontrar durante
    el desarrollo.\\
    
    \item \textbf{Formación}
    
    Al comienzo del proyecto desconocía por completo
    el uso de bibliotecas imprescindibles como \textsc{Ogre3D}, \textsc{OIS} o
    \textsc{MyGUI} y herramientas importantes como \textit{Blender}. Tampoco
    conocía con la necesaria profundidad los fundamentos del desarrollo
    de videojuegos en tres dimensiones. Fue necesaria, por tanto,
    una larga etapa de formación personal utilizando varios recursos bibliográficos
    y pequeñas pruebas prácticas.\\
    
    \item \textbf{IberOgre}
    
    Tras el periodo de formación inicial, dio comienzo el montaje de la wiki
    \wiki. La Oficina de Software Libre de la Universidad de Cádiz (OSLUCA)
    llevó a cabo la instalación de la plataforma sobre sus servidores y,
    desde dicho momento, se pudo comenzar a trabajar en la documentación
    de \textsc{Ogre3D}.
    
    \begin{enumerate}
        \itemsep0em
        \item \textit{Plantillas}
        
        Fue necesario transformar la apariencia que incluye una plataforma
        \textit{Wikimedia} por defecto para conseguir un aspecto propio
        en \wiki. Para ello se modificaron las hojas de estilo \textsc{CSS},
        se trabajó sobre la portada y se crearon varias plantillas. Estas
        plantillas proporcionan una forma sencilla de incluir estructuras
        en varios puntos del sitio como ejemplos, tablas especiales, etc.\\
        
        \item \textit{Sección: Primeros pasos}
        
        Tras preparar la wiki, se procedió a la redacción del artículo
        correspondiente a la primera sección de \wiki. En dicho artículo
        se daba una bienvenida a la documentación y se explicaban sus
        motivaciones, objetivos y estructura en cuanto a lecciones. De esta
        manera, el lector podría conocer la filosofía de trabajo de la plataforma.\\
        
        \item \textit{Sección: \textsc{Ogre3D}}
        
        Posteriormente se comenzó a elaborar los artículos pertenecientes
        a la sección del motor \textsc{Ogre3D} junto a sus ejemplos prácticos. 
        Estos textos cubren prácticamente la totalidad del uso del motor
        en cuanto a aspectos básicos se refiere: instalación, arquitectura,
        inicialización, configuración, creación de escenas, animación, iluminación,
        materiales, efectos de partículas, etc.\\
        
        \item \textit{Sección: Otras tecnologías}
        
        \textsc{Ogre3D} no es un motor de videojuegos completo, sólo es un
        motor de renderizado. Por tanto, el uso de tecnologías complementarias
        es prácticamente obligado. En \wiki\ se han elaborado artículos
        para asistir a los lectores en dichas tecnologías como \textsc{OIS}
        para gestión de dispositivos de entrada o \textsc{SDL mixer} para
        música y efectos de sonido.\\
        
        \item \textit{Sección: Matemáticas}
        
        En la sección de matemáticas se presentan conceptos fundamentales
        para el desarrollo e videojuegos. Sobre todo se cubre de manera
        ligera temas relacionados con la geometría del espacio: puntos,
        vectores, matrices y cuaternos. Siempre se ofrecen aplicaciones
        prácticas dentro del ámbito que nos ocupa.\\
        
    \end{enumerate}
    \item \textbf{Sion Tower}
    
    Tras el periodo de aprendizaje, casi de manera simultánea al trabajo
    en \wiki, se comenzó con el videojuego \juego. El primer paso fue solicitar
    la creación de un nuevo proyecto en la \textit{Forja de RedIRIS} y 
    esperar su aprobación. Dicha forja proporciona un repositorio
    \textit{Subversion} y herramientas web que ayudan en la gestión de
    un proyecto libre: gestor de tareas, subida de ficheros, publicación
    de noticias, etc.
    
    \begin{enumerate}
        \itemsep0em
        \item \textit{Documento de diseño}
        
        En el documento de diseño de un videojuego se detallan elementos
        como la historia, género, personajes, mecánicas de juego, objetos
        o enemigos entre otros muchos. En definitiva, es el documento
        que define de forma más o menos concisa cómo será el videojuego.
        Se trata de un escrito muy importante ya que ayuda a que todo el
        equipo albergue la misma idea sobre el videojuego y pueda trabajar
        de forma más compenetrada.\\
        
        \item \textit{Análisis}
        
        La fase de análisis dio comienzo tras la redacción y revisión del
        documento de diseño. Se procedió con la toma de requisitos a partir
        de dicho documento, se confeccionaron los casos de uso, se elaboró
        el modelo conceptual de datos y se detalló el modelo de comportamiento.\\
        
        \item \textit{Diseño}
        
        Durante la fase de diseño, que siguió de forma inmediata al análisis,
        se elaboraron los diagramasde clases de diseño.\\
        
        \item \textit{Implementación}
        
        La fase de implementación de \juego\ fue, con diferencia, la más
        extendida de todo el desarrollo. Quizás viniese motivada por
        la inexperiencia y por el gran número de subsistemas que se implementaron
        desde cero.\\
        
        \item \textit{Pruebas}
        
        Durante la implementación se fueron realizando pruebas de módulos
        individuales pero fue tras finalizar ficha fase cuando tuvieron
        lugar las pruebas de integración. No sólo se trabajó para que
        el código fuese correcto sino que \juego\ fue probado de forma
        extensiva por colaboradores distintos al desarrollador principal con
        el claro objetivo de pulir ciertos detalles relacionados con el balanceo
        de características (personajes muy débiles o poderosos) o la comodidad
        del control entre otros.\\
        
        \item \textit{Mantenimiento}
        
        Tras reunir sugerencias del público interesado en el proyecto se
        realizaron pequeñas mejoras y ajustes que en pocas ocasiones implicaron
        cambios en el código fuente. Sobre todo, estaban relacionados
        con el balanceo de personajes.\\
        
        \item \textit{Distribución}
        
        Se crearon y publicaron paquetes descargables tanto para sistemas GNU/Linux
        como para Windows. La dificultad que conllevaba empaquetar una
        aplicación \textsc{Ogre3D} en Debian provocó que se tuviera
        que distribuir un paquete con el código fuente en GNU/Linux. Para
        Windows existe un paquete con el ejecutable \texttt{.exe}, los ficheros
        multimedia y las dependencias en forma de bibliotecas dinámicas.\\
        
        \item \textit{Arte}
        
        El proceso de creación de todo el arte necesario para \juego\ comenzó
        nada más acabar la redacción del documento de diseño. Desde ese momento
        se conocía el estilo visual del juego, las pantallas y los personajes
        que aparecerían. Por su complejidad, el trabajo se extendió prácticamente
        durante todo el desarrollo. Al ser la única tarea en la que 
        participaron colaboradores, fueron necesarias labores de supervisión
        y coordinación: estilo, formato de entrega, corrección de fallos, etc.\\
        
    \end{enumerate}
    \item \textbf{Memoria}
    
    Tras el desarrollo de \wiki\ y \juego\ se procedió a la redacción
    del presente documento que incluye apéndices adicionales como el 
    manual de usuario del videojuego.\\
    
    \item \textbf{Presentación}
    
    Como última tarea en la organización temporal figura la elaboración
    de la exposición de cara a la presentación del \textbf{Proyecto fin
    de Carrera}. Se llevó a cabo tratando de plasmar el trabajo realizado
    y los objetivos conseguidos con el desarrollo de este proyecto.\\
    
    \item \textbf{Comunidad}
    
    Una de las partes fundamentales de este proyecto es su objetivo de servir
    a la comunidad. Hasta el momento no existía una documentación extensa
    sobre \textsc{Ogre3D} y el desarrollo de videojuegos tridimensionales
    en castellano. A lo largo de todo el desarrollo ha estado presente la
    atención e interacción con la comunidad a través de diversos medios
    como redes sociales, la propia wiki y el blog.\\
    
    \item \textbf{Presentación (hito)}
    
    Finalmente, el día 1 de septiembre de 2011 se entrega toda la documentación
    del proyecto y se preparó la presentación final que tendría lugar,
    aproximadamente, una semana después.\\
    
\end{enumerate}
