\documentclass[a4paper,11pt]{article}

\usepackage{../memoria/estilos/estiloBase}
\usepackage{../memoria/estilos/colores}
\usepackage{../memoria/estilos/comandos}
\graphicspath{{./imagenes/}}

\title{IberOgre, wiki de Ogre3D en español\\y Sion Tower, videojuego de estrategia}
\author{Alumno: David Saltares Márquez\\Directores: Manuel Palomo Duarte y Francisco Palomo Lozano}
\date{\today}

\begin{document}

\maketitle

% Abstract
\abstract{\noindent
\wiki\ es una documentación libre en formato wiki sobre desarrollo
de videojuegos en tres dimensiones utilizando el motor de renderizado
también libre \textsc{Ogre3D} \cite{junk06}. Incluye documentación relacionada
con tecnología complementaria necesaria para el desarrollo de videojuegos.
Asimismo, abarca los conceptos matemáticos fundamentales.\\

\noindent \juego\ es un videojuego de estrategia y acción en 3D de ambientación
fantástica cuyo objetivo es servir de ejemplo final a la wiki. Hace uso de
\textsc{Ogre3D} y otras tecnologías complementarias documentadas
en \wiki. En \juego\ controlamos a un joven hechicero iniciado que se ve obligado
a defender la Torre Sagrada de una invasión enemiga mientras sus compañeros
están celebrando un rito secreto en el exterior.
}

\tableofcontents

% Contexto y motivaciones
\section{Introducción}

\subsection{Motivaciones}

\noindent Comenzar a utilizar una tecnología desconocida como \textsc{Ogre3D} o cualquier
otra siempre entraña cierta dificultad y, si le añadimos la barrera del 
idioma, aún más. La principal motivación para embarcarme en la confección
de una documentación sobre desarrollo de videojuegos en 3D como es \wiki\
ha sido la completa ausencia de información al respecto en castellano.
No existía una plataforma de aprendizaje y documentación equivalente a
la wiki oficial de \textsc{Ogre3D} en español.\\

\noindent \textbf{Diseño de Videojuegos} es una asignatura optativa de tercer curso
de la titulación Ingeniería Técnica en Informática de Sistemas dentro
de la Universidad de Cádiz. En dicha asignatura los alumnos se organizan
en grupos de tres con el objetivo de desarrollar un juego sencillo en dos
dimensiones durante el cuatrimestre. Los alumnos de la
asignatura suelen utilizar bibliotecas libres como \textsc{libSDL} sobre la que ya se publicó
una excelente documentación en castellano en formato wiki llamada \textbf{Wikijuegos} \cite{website:wikijuegos}.
Una vez finalizada la asignatura, podrían ampliar
sus conocimientos gracias a documentación sobre desarrollo de videojuegos en
3D en su idioma.\\

\noindent Mi interés en aprender a desarrollar videojuegos en tres dimensiones fue el
motor principal que me llevó a desarrollar \juego. Son tantos los conceptos
nuevos, las tecnologías desconocidas y las aproximaciones diametralmente
opuestas que las posibilidades de adquirir conocimiento son enormes. En cualquier
caso, \wiki\ necesitaba un gran ejemplo final que pusiera en práctica todo
lo visto a lo largo de las lecciones. Un ejercicio lo más cercano
a la realidad y completamente documentado de manera que cualquier lector pudiera
comprender de primera mano cómo se desarrolla un videojuego en 3D.\\

\noindent \textsc{Ogre3D} es un motor de renderizado gráfico muy potente
y versátil, de ahí su elevada curva de aprendizaje. Cabe destacar que se utiliza
constantemente en videojuegos comerciales como Victory (ver figura
\ref{fig:victory}) y otras simulaciones. Podemos afirmar que es una
tecnología profesional y utilizarla me haría adquirir destrezas importantes
con posibles salidas en el plano laboral de la industria.\\

\figura{victory.jpg}{scale=0.25}{Victory (Windows - 2011)}{fig:victory}{H}

% Objetivos
\subsection{Objetivos}

\noindent \wiki\ debe cumplir con los siguientes requisitos básicos:

\begin{itemize}
    \item Cubrir los principales subsistemas de \textsc{Ogre3D}.
    \item Ofrecer los contenidos matemáticos mínimos sobre geometría del
    espacio.
    \item Incluir una sección para otras tecnologías complementarias a \textsc{Ogre3D}
    y necesarias para el desarrollo de videojuegos.
    \item Emplear una aproximación lo más práctica posible.
    \item Prestar especial atención y soporte a herramientas de desarrollo
    libres y multiplataforma.
    \item Ofrecer técnicas y buenas prácticas para el desarrollo de software
    multiplataforma.
    \item Construir una comunidad que colabore de forma activa con la
    documentación probado ejemplos, informando de errores encontrados y
    creando nuevo contenido de interés para el resto de usuarios.
\end{itemize}

\noindent \juego\ debe cumplir con los siguientes requisitos:

\begin{itemize}
    \item Servir de ejemplo lo más cercano a la realidad posible de un
    desarrollo de videojuego 3D.
    \item Crear un pequeño motor orientado a la producción de contenido.
    Cualquier diseñador podría añadir niveles al juego sin necesidad
    de tocar el código fuente ni recompilarlo.
    \item Convertirse en un ejemplo de trabajo con un equipo multidisciplinar.
    \item \juego\ debe ser cómodo de utilizar, intuitivo y divertido.
    \item Crear subsistemas independientes del videojuego reutilizables
    por la comunidad de usuarios.
\end{itemize}

% Organización temporal
\section{Organización temporal}

\noindent La planificación del desarrollo se ha llevado a cabo utilizando
la herramienta \textit{Planner} \cite{website:planner}. Como puede observarse en el diagrama de Gantt,
en tareas relacionadas con el apartado artístico de \juego\ participan
más personas. He contactado con expertos en materias artísticas dispuestos
a colaborar en el desarrollo de un videojuego completamente libre. Finalmente,
los participantes adicionales son:

\begin{itemize}
    \itemsep0em
    \item Antonio Jiménez Rodríguez: artista 3D encargado de diseñar, modelar,
    texturizar al protagonista y a los enemigos del juego.
    \item Daniel Belohlavek: artista 2D encargado de los iconos de los hechizos.
    \item Daniel Pellicer García y Antonio Caro Oca: encargados de componer
    y producir la banda sonora completa del juego.
    \item Francisco Martín Márquez: técnico de sonido encargado de grabar
    y procesar los efectos de sonido para el juego.
\end{itemize}

\noindent Por cuestiones de espacio y formato, el diagrama de Gantt ha sido divido
en dos. En la figura \ref{fig:planificacion1} puede observarse la planificación
del proyecto correspondiente al intervalo comprendido entre julio de 2010
y enero de 2011. Por otro lado, la figura \ref{fig:planificacion2} muestra
el resto de la planificación, desde febrero de 2011 hasta septiembre de 2011.

\figura{planificacion1.jpg}{scale=0.35}{Planificación del proyecto desde julio de 2010 hasta enero de 2011}{fig:planificacion1}{H}

\figura{planificacion2.jpg}{scale=0.35}{Planificación del proyecto desde febrero de 2011 hasta septiembre de 2011}{fig:planificacion2}{H}

% Desarrollo de IberOgre
\section{Desarrollo de IberOgre}

\noindent En esta sección trataremos los aspectos más relevantes del
desarrollo de la plataforma educativa \wiki. Por razones de espacio, se
obvian muchos detalles que sí aparecen en la memoria de Proyecto fin de Carrera.
\wiki\ \cite{website:iberogre} utiliza el motor MediaWiki \cite{website:mediawiki}
y ha sido publicada en los servidores de la
Oficina de Software Libre de la Universidad de Cádiz.\\

\url{http://wikis.uca.es/iberogre}\\

\subsection{Usuarios}

\noindent No todos los usuarios que accedan a \wiki\ tendrán los mismos privilegios.
Existirán tres categorías de usuarios: lector, usuario registrado y administrador.
Los primeros tendrán acceso de lectura a todo el contenido, los segundos
podrán hacer ediciones y crear nuevos artículos sin alterar la estructura general.
El administrador, además podrá hacer cambios en la estructura, gestionar otros
usuarios así como hacerse cargo de las extensiones de la plataforma.\\

\noindent Cualquier persona que acceda al sitio web de la plataforma será considerado
como lector no registrado. En el momento en el que rellene el formulario
y envíe sus datos se convertirá en usuario registrado. Lo habitual es que
sólo exista un administrador del sistema o exista un grupo reducido de
administradores.\\

\subsection{Secciones y estructura de los artículos}

% Bloques temáticos
\noindent Los artículos de \wiki\ están organizados en varios bloques temáticos
para abarcar todos los objetivos que nos hemos marcado.
Esta organización también busca facilitar la navegación
al usuario. Los bloques principales de \wiki\ son:\\

\begin{itemize}
    \itemsep0em
    \item \textbf{Primeros pasos}: bloque introductorio en el que el lector
    debe poder conocer la estructura, filosofía y metodología de la plataforma
    de aprendizaje.
    \item \textbf{Programación de videojuegos 3D}: sección que se dedicará
    a ofrecer los conocimientos matemáticos mínimos para desarrollar
    videojuegos en 3D. Se centrará en conceptos generales de geometría
    del espacio y álgebra lineal desde una perspectiva eminentemente práctica.
    \item \textbf{Ogre3D}: bloque central que desglosa los apartados más
    relevantes del motor de renderizado \textsc{Ogre3D}. Lo más
    importante es que se comprenda el funcionamiento del motor y se sea
    capaz de ampliar conocimientos de forma autónoma.
    \item \textbf{Otras tecnologías}: enfocada a introducir en el uso de
    tecnologías complementarias a \textsc{Ogre3D} como \textsc{OIS} (entrada),
    \textsc{libSDL mixer} (audio) u \textsc{OgreBullet} (físicas).
    \item \textbf{Videojuegos}: en esta sección se adjuntará documentación
    sobre el desarrollo de videojuegos libres que hagan uso de \textsc{Ogre3D}.
    \juego\ contará con el primer artículo de esta sección.
\end{itemize}

\noindent La estructura lógica de los artículos queda reflejada en la figura \ref{fig:bloques-articulos}
y está pensada para que el lector adquiera de la manera más sencilla posible
los conocimientos necesarios para continuar. En primer lugar se hace una
introducción para que el lector sepa qué esperar del artículo. Posteriormente
se listan los requisitos previos para que el usuario pueda adquirirlos y comprender
el desarrollo. Más adelante se desglosa el tema intercalando explicaciones teóricas
y pequeños ejemplos. Después se adjunta un ejemplo descargable profusamente documentado
que ilustra todo lo visto en el texto. Finalmente se incluye un pequeño párrafo
a modo de conclusión.\\

\figura{bloques-articulos.png}{scale=0.5}{Estructura de los artículos en IberOgre}{fig:bloques-articulos}{h}

\subsection{Artículos publicados} 

\noindent La sección \textit{Primeros pasos} únicamente cuenta con un artículo
introductorio.

\begin{itemize}
    \itemsep0em
    \item Comenzando en IberOgre
\end{itemize}

\noindent En el bloque \textit{Programación de videojuegos 3D} se encuentran tres
artículos.

\begin{itemize}
    \itemsep0em
    \item Introducción, puntos y vectores
    \item Matrices
    \item Cuaternos
\end{itemize}

\noindent En el bloque principal, \textit{Ogre3D} tenemos los siguientes artículos.

\begin{itemize}
    \itemsep0em
    \item Conociendo Ogre3D
    \item Conceptos generales
    \item Instalación de Ogre3D 1.7 en GNU/Linux
    \item Instalación de Ogre3D 1.7 en Windows
    \item Creación de un entorno de trabajo multiplataforma
    \item Inicialización y cierre de Ogre3D
    \item Gestión de recursos
    \item Creación básica de escenas
    \item Materiales
    \item Manipulación de nodos
    \item Luces, sombras y entorno
    \item Animación
    \item Sistemas de partículas
    \item Sistema de Overlays
\end{itemize}

\noindent Dentro de la sección \textit{Otras tecnologías} tenemos los siguientes
artículos.

\begin{itemize}
    \itemsep0em
    \item Manejo básico de OIS
    \item Exportar modelos desde Blender
    \item Colisiones y físicas con OgreBullet
    \item Extender la gestión de recursos, audio
\end{itemize}

\figura{ejemplo-audio.jpg}{scale=0.2}{Ejemplo de extensión del sistema de recursos, audio}{fig:ejemplo-audio}{h}

% Desarrollo de Sion Tower
\section{Desarrollo de Sion Tower}

\noindent \juego\ es un videojuego de estrategia y acción en el que controlamos a
un joven mago iniciado llamado \prota. Nuestro protagonista es el
único que permanece en la Torre Sagrada mientras sus compañeros han salido
a celebrar un rito y es entonces cuando unos monstruos invaden la Torre.
\prota\ debe proteger el recinto Sagrado a toda costa eliminando
a todos los enemigos lanzando hechizos administrando sabiamente su energía
mágica. Cada nivel es un piso de la Torre.\\

\subsection{Descripción general}

\figura{siontower-menu.jpg}{scale=0.2}{Menú principal de Sion Tower}{fig:siontower-menu}{H}

% Sistema de juego
\noindent En \juego\ sólo existe una modalidad para un jugador individual
que sigue la trama del juego y la historia de la invasión enemiga. No obstante,
podemos contar con varios perfiles de jugador identificados por un nombre
para que varios usuarios compartan el juego en una misma máquina. En la
figura \ref{fig:selperfil} puede observarse la pantalla de selección de perfil.
Inicialmente, para cada perfil solo se encuentra desbloqueado el primer nivel.
Para avanzar en la historia debemos completar con éxito los niveles de
forma que el siguiente se desbloquee.\\\\

\figura{siontower-perfil.jpg}{scale=0.2}{Pantalla de selección de perfil en Sion Tower}{fig:selperfil}{h}

\noindent Una vez seleccionamos un nivel comienza la partida y vemos situado al protagonista.
Progresivamente comienzan a llegar oleadas de enemigos que buscarán a \prota\
para atacarle. Podemos mover al personaje y controlar la cámara para situarnos
según nos convenga (ver manual de usuario). \prota\ cuenta con energía vital
y mágica limitadas, cuando se le acabe la vital habremos perdido la partida
y si carece de mágica no podremos lanzar hechizos. Existen varios tipos de hechizos
con distinto poder y coste en energía mágica, es necesario saber administrarlo correctamente.

\figura{siontower-juego.jpg}{scale=0.20}{Partida de Sion Tower}{fig:juego}{H}

\noindent Si eliminamos a todos los enemigos pasaremos a la pantalla de celebración
en la que desbloqueamos un nuevo nivel y adquirimos experiencia. La experiencia
es una puntuación que se incrementa en función de los enemigos eliminados,
la vida restante, el maná utilizado y el tiempo invertido en finalizar el nivel.
Desde dicho punto podemos volver a la selección de nivel para probar la
siguiente prueba.\\

% Personajes

\noindent \juego\ cuenta con varios enemigos diferentes: goblin, diablillo
y gólem de hielo. El goblin es una criatura muy básica y débil que se aprovecha
de la superioridad numérica para atacar. Los diablillos son seres más veloces
y poderosos pero escasos. Los gólems de hielo son muy lentos pero un golpe
suyo puede causar mucho daño.\\

\figura{goblin.jpg}{scale=0.2}{Goblin, enemigo de Sion Tower}{fig:goblin}{H}

% Hechizos

\noindent Existen tres tipos de hechizos en \juego. Con \textbf{Bola de fuego} puedes
invocar un proyectil mágico que abrase a tus enemigos a su paso. 
\textbf{Furia de Gea} convoca la propia fuerza de la naturaleza para volverla
en contra de cualquier criatura. Finalmente, \textbf{Ventisca} lanza varios
proyectiles mágicos helados y afilados como cuchillas.\\

\subsection{Lenguajes, bibliotecas y herramientas}

\noindent En esta sección listaremos los distintos lenguajes, bibliotecas
y herramientas auxiliares empleadas en el desarrollo de \juego. Se han
utilizado los siguientes lenguajes:

\begin{itemize}
    \item \textbf{C++}: lenguaje de programación utilizado para todo el
    código del juego.
    \item \textbf{Python}: lenguaje de scripting orientado a objetos empleado
    en pequeñas herramientas auxiliares.
    \item \textbf{XML}: lenguaje de marcado empleado para describir plantillas
    de interfaz y escenas tridimensionales.\\
\end{itemize}

Han sido utilizadas las siguientes bibliotecas.

\begin{itemize}
    \item \textbf{Ogre3D}: motor de renderizado 3D libre empleado tanto
    en \wiki\ como en \juego.
    \item \textbf{MyGUI}: biblioteca complementaria a \textsc{Ogre3D} para
    crear y gestionar interfaces de usuario complejas.
    \item \textbf{libSDL mixer}: biblioteca para reproducir efectos de sonido
    y piezas musicales.
    \item \textbf{Boost}: biblioteca que extiende las funcionalidades de C++
    de una forma complementaria a como lo hace la \textsc{STL}.
    \item \textbf{pugixml}: biblioteca muy eficiente para procesar ficheros
    XML.\\
\end{itemize}

Se ha hecho uso de las siguientes herramientas.

\begin{itemize}
    \item \textbf{Subversion}: sistema de control de versiones.
    \item \textbf{GNU GCC Compiler}: compilador de C++ del proyecto GNU.
    \item \textbf{Make}: automatización del proceso de compilación.
    \item \textbf{GNU Debugger}: depurador del proyecto GNU.
    \item \textbf{Valgrind}: depuración del uso de la memoria.
    \item \textbf{Vim}: editor de textos.
    \item \textbf{\LaTeX}: lenguaje de marcado y creación de documentos.
    \item \textbf{Doxygen}: documentación automática de código.
    \item \textbf{Blender}: modelado, texturizado y animación 3D.
    \item \textbf{GIMP}: software de diseño gráfico 2D.
    \item \textbf{Inkscape}: software de creación de gráficos vectoriales 2D.
    \item \textbf{MyGUI Layout Editor}: editor visual de plantillas de interfaces para \textsc{MyGUI}
    \item \textbf{Particle Editor}: editor de sistemas de partículas.
    \item \textbf{Audacity}: editor de audio.
    \item \textbf{XvidCap}: capturador de vídeo para las muestras de los avances en \juego.
    \item \textbf{OpenShot Video Editor}: editor de vídeo para montar los
    tráilers de \juego.
    \item \textbf{Planner}: herramienta para planificación de proyectos.
    \item \textbf{BOUML}: editor de diagramas con notación UML.
    \item \textbf{Dia}: herramienta de creación de diagramas con varias notaciones.\\
\end{itemize}

\subsection{Implementación}

\noindent En esta sección describiremos brevemente los problemas de implementación
de mayor relevancia a los que me he enfrentado durante el desarrollo de \juego. En la memoria
del Proyecto fin de Carrera puede obtenerse documentación mucho más detallada
sobre la manera de resolver dichos problemas que, por razones de espacio,
es imposible incluir en el resumen. A la hora de diseñar e implementar
los distintos subsistemas se ha recurrido en alta medida a las publicaciones
\textit{Fundamentos de C++} \cite{gera09},
\textit{Design Patterns} \cite{gamm77} y \textit{Game Engine Architecture}
\cite{greg09} entre otras indicadas en los apartados correspondientes.\\

\begin{description}
    \item [Estados de juego] en el videojuego contamos con varias pantallas
    entre menús y la propia partida por lo que es necesario modelarlas
    de forma efectiva. Cada estado debe capturar los eventos que se producen
    (pulsación de teclas, botones del ratón y movimiento del ratón) para
    darles respuesta. Además, debe poder actualizarse de forma lógica en
    cada iteración del bucle de juego. Es altamente recomendable que
    las transiciones entre estados se manejaran de forma segura para no
    producir problemas de memoria. Por supuesto, cada estado debe
    controlar que los elementos que contiene no produzcan pérdidas de memoria.\\
    
    \item [Internacionalización mediante gettext] \juego\ está completamente
    internacionalizado mediante la conocida biblioteca \textsc{gettext} \cite{pdf:jtgettext}.
    Este sistema emplea duplas clave valor para las cadenas traducibles y
    sus traducciones al idioma destino. Con \textsc{gettext} es posible
    extraer todas las cadenas de código C++ pero no ocurre así con el texto de las
    plantillas de interfaz de \textsc{MyGUI}. Se ha desarrollado
    un script en Python para realizar dicha tarea por nosotros de forma automatizada.\\
    
    \item [Sistema de audio] \textsc{Ogre3D} no incluye un sistema de audio
    por lo que era necesario recurrir a otra biblioteca como \textsc{libSDL mixer}.
    Se ha decidido extender el sistema de gestión de recursos del motor
    de renderizado para que soporte dos nuevos recursos: efectos de sonido
    y piezas musicales.\\
    
    \item [Detección de colisiones] en el juego hay muchos elementos cuyas
    colisiones debemos detectar: protagonista, enemigos, escenario y hechizos.
    Era imprescindible un sistema de detección de colisiones que soportara
    varias formas geométricas, extensible, altamente reutilizable (para publicarlo
    de forma independiente) y eficiente para soportar varios elementos
    en la escena. Se ha utilizado como referencia para los tests de colisión
    entre varias formas geométricas el libro \textit{Real Time Collision
    Detection} \cite{eric05}.\\
    
    \item [Exportación de modelos 3D] el artista 3D de \juego\ utiliza la herramienta
    \textit{Cinema4D} para el modelado y animación de personajes. Existe un
    plugin para exportar el formato a \textsc{Ogre3D} pero es privativo y no
    corrige de forma correcta las escalas de los modelos (para mantener una
    relación de tamaños uniforme). Se ha codificado un script en Python
    que corrije dicho problema tanto en mallas tridimensionales como en
    esqueletos.\\
    
    \item [Creación de niveles con Blender] uno de los objetivos del juego
    era crear un motor orientado a la producción de contenido por lo que
    era necesario un sistema por el cual los usuarios pudieran añadir nuevos
    niveles a \juego. Siguiendo las instrucciones del manual de usuario
    es posible crear un nivel con la herramienta 3D \textit{Blender} y
    exportarlo al formato de descripción de escenas en XML \textit{Dotscene}.
    El motor del juego debe procesar dicho fichero y obtener información como:
    elementos del escenario, posición del personaje, oleadas de enemigos,
    malla de navegación y elementos colisionables.\\
    
    \item [Búsqueda de caminos] es necesario que los enemigos \textit{vean}
    para poder encontrar el camino hacia el personaje con el objetivo de
    atacarle sin que colisionen con obstáculos por el camino. Se hace uso
    de una malla de navegación formada por triángulos interconectados a modo
    de grafo conexo. Ha sido imprescindible encontrar un algoritmo de
    búsqueda de caminos mínimos que trabaje sobre dicho grafo de forma
    eficiente ya que estamos ante una aplicación en tiempo real.\\
    
    \item [Algoritmos de movimiento] los enemigos deben moverse por el camino
    obtenido tras la búsqueda de forma natural y realista. Para ello
    empleamos algoritmos de movimiento dinámicos o \textit{Steering
    Behaviors} \cite{mill09}. Debemos modelar distintos tipos de comportamientos:
    seguir un camino de puntos, evitar colisionar con otros compañeros,
    perseguir a un personaje, llegar a un destino, etc.\\

\end{description}


% Conclusiones
\section{Conclusiones}

\subsection{Conclusiones personales}

\noindent Con \wiki\ y \juego\ he adquirido muchos conocimientos y me he
enriquecido enormemente como persona y desarrollador. Si bien ya había
trabajado de forma independiente en algún videojuego sencillo o proyecto
de distinta índole, este sin duda era el reto de mayor envergadura al que
me había enfrentado jamás. De ahí que la fase de aprendizaje fuese tan
larga al comienzo del proyecto aunque después se extendiera durante
todo este tiempo.\\

\noindent En los siguientes puntos repasaré de forma superficial lo que he aprendido:

\begin{enumerate}
    \itemsep0em
    \item \textbf{Juegos 3D}: hasta el momento había desarrollado juegos
    sencillos en dos dimensiones pero no conocía el mundo de las tres dimensiones.
    Consideré que el Proyecto Fin de Carrera me brindaba una oportunidad
    excelente para aprender y me decidí a intentarlo. Es una aproximación
    muy diferente, atractiva y llena de retos.
    
    \item \textbf{Nuevas bibliotecas}: he hecho uso de muchas bibliotecas
    que no conocía y he necesitado aprender a utilizarlas correctamente.
    Entre estas bibliotecas se encuentran la propia \textsc{Ogre3D},
    \textsc{MyGUI}, \textsc{Boost} o \textsc{pugixml}. Anteriormente
    había utilizado \textsc{libSDL mixer} pero nunca me había visto obligado
    a integrarla dentro de un sistema más grande como es la gestión
    de recursos de \textsc{Ogre3D}.
    
    \item \textbf{Matemáticas para videojuegos}: no sólo he tenido que repasar
    conceptos matemáticos de geometría del espacio y álgebra sino que he
    tenido que incorporar otros nuevos. Tanto de cara a \wiki\ como para
    \juego\ me he visto obligado a aplicar estos conceptos con el objetivo
    de buscar soluciones a problemas de programación.
    
    \item \textbf{Lenguaje C++}: desde el segundo curso de Ingeniería
    Técnica en Informática de Gestión he estado utilizando C++ no sólo
    dentro del ámbito estrictamente académico. No obstante, desarrollar
    \juego\ me ha servido para profundizar en los detalles del lenguaje
    y aprender a utilizarlos a mi favor.
    
    \item \textbf{Python}: este lenguaje de scripting orientado a objetos
    se ha mostrado extremadamente útil a la hora de desarrollar
    pequeños programas auxiliares. Por ejemplo, lo he utilizado en el script
    que soluciona el problema de la escala en la exportación de modelos
    tridimensionales y en el que extrae las cadenas traducibles de plantillas
    de interfaz. Es un lenguaje extremadamente sencillo y fácil de aprender
    pero muy potente \cite{pilgr04}.
     
    \item \textbf{Optimización}: los videojuegos son sistemas complejos
    que, de no prestar especial atención, podrían hacer un consumo irracional
    de recursos. Es necesario aplicar técnicas de optimización tanto en
    términos de uso de procesador como de consumo de memoria. Este proyecto
    me ha servido para conocer varias de estas técnicas y algoritmos concretos
    cuyo rendimiento es superior al de otros en momentos concretos. Por ejemplo,
    en primer lugar empleé el algoritmo A* para la búsqueda de caminos
    aunque la precomputación con Floyd probó ser más eficiente.
    
    \item \textbf{Técnicas de IA}: hasta el momento conocía algunas técnicas
    de inteligencia artificial sobre teoría de juegos, búsqueda de caminos
    o autómatas para modelar comportamientos. No obstante, no había aplicado
    dichas técnicas al mundo de las tres dimensiones. En \juego\ se realiza
    una búsqueda de caminos a partir de una malla definida en tiempo de diseño.
    Además, los enemigos hacen uso de algoritmos de movimiento
    (\textit{Steering Behaviors}) que desconocía hasta este momento.
     
    \item \textbf{Diseño de un videojuego}: la redacción de un documento de diseño
    en el que se detallara la forma de jugar, personajes, historia, etc
    ha sido muy necesaria ya que ha ayudado al proceso de análisis
    y diseño. Además, los colaboradores han podido conocer las necesidades
    del proyecto en cuanto a recursos artísticos se refiere.
    
    \item \textbf{Trabajo en equipo}: en \wiki\ se ha trabajado junto
    a la comunidad en todo momento. Los lectores enviaban sus opiniones y
    éstas debían ser tenidas en cuenta. En \juego\ el trabajo en equipo
    se hizo mucho más evidente ya que colaboramos en todo momento seis
    personas especializadas en disciplinas muy distintas. Hubo que realizar
    labores de coordinación, comunicación y resolución de conflictos.
    
    \item \textbf{Redacción en MediaWiki}: hasta el momento mi relación
    con el motor \textit{MediaWiki} se había limitado a ediciones muy
    esporádicas en Wikipedia que no requerían conocimientos de la sintaxis
    que se utilizaba. Me he visto obligado a conocer la sintaxis de redacción
    en \textit{Wikimedia} para redactar correctamente los artículos.
    
    \item \textbf{Trabajo en wikis}: no sólo basta con conocer la sintaxis
    de \textit{Wikimedia} para poder participar en la confección de una wiki.
    Existen toda una serie de convenciones de escritura, comportamiento,
    nomenclatura y estructuras que deben ser conocidas.
    
    \item \textbf{Aplicación de conocimientos}: el desarrollo de este
    proyecto me ha sido especialmente útil para poner en práctica aquellos
    conocimientos adquiridos durante la Ingeniería Técnica. Sobre todo
    me refiero a aquellos relacionados con la Ingeniería del Software.
\end{enumerate}

\subsection{Conclusiones técnicas}

\noindent En \wiki\ y \juego\ se han cumplido con los objetivos propuestos en el
capítulo introductorio de esta memoria de Proyecto Fin de Carrera. En la
plataforma de aprendizaje se ha conseguido:

% Objetivos cumplidos con IberOgre
\begin{itemize}
    \itemsep0em
    \item Se ha creado contenido organizado en bloques temáticos tal y como
    se propuso: introducción, matemáticas para videojuegos, \textsc{Ogre3D},
    otras tecnologías y videojuegos desarrollados con \textsc{Ogre3D}.
    \item La navegación es intuitiva y los artículos están ordenados
    de forma aproximada en dificultad ascendente.
    \item Es posible adquirir los conocimientos de geometría del espacio
    necesarios para desarrollar juegos en 3D.
    \item Se han cubierto los aspectos básicos del uso del motor de renderizado
    \textsc{Ogre3D}.
    \item Otras tecnologías enfocadas al desarrollo de videojuegos han sido
    explicadas a través de varios artículos.
    \item Los artículos están dotados de un enfoque práctico gracias a los
    ejemplos finales y a los pequeños fragmentos de código intermedios.
    \item Varios usuarios han mostrado interés, han colaborado con el proyecto
    ya sea mediante correcciones, artículos, sugerencias o ayudando a difundir
    la plataforma.
\end{itemize}

% Objetivos cumplidos con Sion Tower
\noindent En el videojuego \juego\ hemos conseguido:

\begin{itemize}
    \itemsep0em
    \item Construir un videojuego completo multiplataforma empleando los conocimientos de
    \wiki\ y otros conceptos profusamente documentados.
    \item Cuatro niveles completos.
    \item Tres tipos de enemigos distintos y tres tipos de hechizos diferentes.
    \item Iluminación y sombras dinámicas.
    \item Alto rendimiento incluso en equipos modestos.
    \item Efectos especiales avanzados utilizando sistemas de partículas.
    \item Motor orientado a la creación de contenido, es posible
    añadir niveles con \textit{Blender}, una herramienta ampliamente
    utilizada en el mundo profesional de diseño, publicidad y videojuegos.
    \item Internacionalización completa gracias a \textsc{gettext} y scripts auxiliares.
    \item Creación de un videojuego que entretiene gracias a las impresiones
    que han prestado los colaboradores a lo largo de todas las fases del
    desarrollo.
    \item Implementación de un motor modular fácilmente ampliable. Varios
    subsistemas han sido liberados de forma independiente y están siendo
    utilizados por otros usuarios.
\end{itemize}

\noindent Se han generado estadísticas sobre el uso del repositorio \textit{Subversion}
empleando la herramienta libre \textit{StatsSVN}.
En total el proyecto está compuesto por 619.000 líneas de código aunque
entre ellas se encuentran las distintas ramas (con código duplicado)
y líneas de documentación (esta memoria, por ejemplo). Cabe destacar
que se han realizado más de 500 commits, lo que permite volver hacia atrás
de forma sencilla. Es posible acceder al informe de estadísticas desde
la siguiente dirección.\\

\noindent \url{http://siondream.com/iberogre-siontower-statsvn}\\

\subsection{Posibles mejoras}

\noindent Es cierto que los objetivos que nos marcamos al comienzo del desarrollo
han sido cumplidos satisfactoriamente, no obstante se han detectado puntos
en los que el proyecto podría mejorar. A continuación hacemos una lista
de las posibles mejoras de \wiki.

% Mejoras en IberOgre
%   - Artículos en Ogre
%   - Artículos en otras tecnologías
%   - Artículos en videojuegos
\begin{itemize}
    \item Consolidación de una comunidad de redactores y lectores para
    la plataforma de aprendizaje.
    \item Nuevos artículos sobre videojuegos desarrollados con \textsc{Ogre3D}
    en los que el propio desarrollador comente la experiencia del desarrollo
    y proporcione o enlace documentación de interés.
    \item Artículos en la sección \textsc{Ogre3D} que documenten los sistemas
    de \textit{shading} con los que cuenta el motor.
    \item Documentar alguna biblioteca o sistema para incluir juego en red en
    la sección de otras tecnologías.\\
\end{itemize}

\noindent En \juego\ podrían incluirse las siguientes mejoras:

% Mejoras en Sion Tower
%   - Más hechizos
%   - Experiencia, subida de nivel, aumento tde características
%   - Idiomas adicionales
%   - IA distinta para cada enemigo

\begin{itemize}
    \item Nuevos hechizos: paralización, teletransporte para huir, colocación
    de trampas, etc.
    \item Mayor relevancia de los puntos de experiencia, podrían permitir
    subir de nivel, mejorar la energía y el maná así como desbloquear
    hechizos.
    \item Enemigos adicionales como arañas gigantes o criaturas voladoras
    con comportamientos diferentes.
    \item Nuevos niveles de mayor tamaño que permitan la aplicación de
    técnicas de búsqueda de caminos jerárquicas.
    \item Escenas narrativas mediante animaciones que fueran contando
    la historia entre niveles.
\end{itemize}


%  - V CUSL
\subsection{V Concurso Universitario de Software Libre}

\noindent El proyecto \wiki\ y \juego\ ha participado en el V Concurso Universitario
de Software Libre \cite{website:cusl} en las categorías de \textit{Comunidad} y \textit{Educación y ocio}.
Se trata de un concurso de software, hardware y documentación libre a nivel
nacional en el que pueden participar grupos de hasta tres estudiantes universitarios,
de bachiller o de ciclos superiores. Se valora el desarrollo del proyecto
desde el comienzo del curso hasta la fase final que tuvo lugar el día
12 de mayo de 2011.\\

\noindent En esta edición se presentaron un total de 115 proyectos de diversa índole
y los resultados que obtuvo \wiki\ y \juego\ no pudieron ser más satisfactorios.
En la fase local del concurso fue galardonado con el premio al mejor
proyecto de \textit{Ocio} mientras que en la fase nacional, que se celebró en Granada,
se recibió el premio al mejor proyecto de \textit{Comunidad}.\\

\noindent El concurso se ha mostrado como un aliciente de lo más positivo para desarrollar
el proyecto con más entusiasmo y apertura hacia la comunidad. Ha sido muy
beneficioso en términos de audiencia y difusión gracias a los medios
que se han hecho eco de la convocatoria. Sin duda, ha sido uno de los
factores que más me ha motivado a seguir hacia delante tomándomelo como un
reto personal y buscando una experiencia enriquecedora junto al resto
de participantes.\\

\figura{vcusl.jpg}{scale=0.35}{Finalistas del V Concurso Universitario de Software Libre}{fig:vcusl}{h}

\section{Comunidad y difusión}

\noindent \wiki\ y \juego\ cuenta con un elevado aspecto de comunidad ya que juntos
forman una plataforma de aprendizaje de desarrollo de videojuegos en tres
dimensiones con \textsc{Ogre3D}. Era imprescindible llevar a cabo acciones
para difundir el proyecto y buscar la creación de dicha comunidad. La comunicación
con los lectores debía ser fluida, directa y cercana tratando de apelar
a su curiosidad e interés por la materia. A continuación, se listan los medios
empleados para contribuir a la difusión del proyecto.\\

\begin{itemize}
    \item \textbf{Blog de desarrollo}: noticias y documentación, se han
    publicado más de 70 artículos y se han recibido más de 85.000 visitas.

    \url{http://siondream.com/blog/category/proyectos/pfc}
    
    \item \textbf{Forja}: se ha hecho uso del repositorio \textit{Subversion},
    la lista de tareas, noticias y sección de ficheros con más de 1.200 descargas.
    
    \url{https://forja.rediris.es/projects/cusl5-iberogre}
    
    \item \textbf{Twitter}: noticias breves y contacto cercano con la comunidad,
    contamos con casi 100 seguidores. A través de este medio el propio Steve Streeting,
    creador de \textsc{Ogre3D}, recomendó el proyecto,
    
    \url{http://twitter.com/#!/IberOgre}
    
    \item \textbf{Web en la forja}: web estática con excelente posicionamiento
    en buscadores que proporciona la forja. Sirve de lista de enlaces a otros
    medios.
    
    \url{http://cusl5-iberogre.forja.rediris.es}
    
    \item \textbf{Canal de Youtube}: 10 vídeos con los avances de \juego\
    que suman más de 2.800 reproducciones.
    
    \url{http://www.youtube.com/user/davidsaltares}
    
\end{itemize}

\noindent El proyecto ha aparecido en varias comunidades de desarrollo hispanohablantes,
ha sido recomendado por el creador de \textsc{Ogre3D} (Steve Streeting) y
ha llegado a la portada de la web oficial. Gracias al V Concurso Universitario
de Software Libre varios periódicos como La Voz o Diario de Cádiz han incluido
artículos al respecto.\\

\section{Licencias libres}

\noindent Toda la documentación de este proyecto, incluyendo este texto y a \wiki\ al
completo, está sujeta a la licencia de documentación GFDL 1.2. Por su
parte, el código fuente de los ejemplos y de \juego\ están liberados
bajo la licencia GPL v3. Los recursos multimedia (música, efectos de sonido,
arte 3D y arte 2D) han sido liberados bajo Creative Commons 3.0 atribución,
no comercial y compartir igual.\\

\noindent Todas las bibliotecas empleadas como son: (\textsc{Ogre3D}, \textsc{MyGUI},
\textsc{libSDL}, \textsc{libSDL mixer}, \textsc{pugixml} y \textsc{OIS}) están disponibles
bajo licencias libres. De esta manera podemos liberar también el proyecto
sin entrar en incompatibilidades con dichas licencias.\\

\noindent Liberar una plataforma educativa como es \wiki, era la decisión
natural. Así, cualquier usuario podría acceder a todo el contenido sin
ningún tipo de restricciones a la hora de pensar en posibles reutilizaciones
o modificaciones. Si se pretendía que el proyecto gozase de una distribución
lo más amplia posible, era necesario eliminar barreras sobre su uso.\\

\clearpage
\phantomsection
\addcontentsline{toc}{section}{Bibliografía y referencias}
\bibliographystyle{is-plain}
\bibliography{bibliografia}

\end{document}
