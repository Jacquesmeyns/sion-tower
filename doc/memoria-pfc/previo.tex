% -*-previo.tex-*-
% Este fichero es parte de la plantilla LaTeX para
% la realización de Proyectos Final de Carrera, protejido
% bajo los términos de la licencia GFDL.
% Para más información, la licencia completa viene incluida en el
% fichero fdl-1.3.tex

% Copyright (C) 2009 Pablo Recio Quijano 

\section*{Agradecimientos}

Me gustaria agradecer y/o dedicar este texto a ...

\cleardoublepage

\section*{Licencia} % Por ejemplo GFDL, aunque puede ser cualquiera

Este documento ha sido liberado bajo Licencia GFDL 1.3 (GNU Free
Documentation License). Se incluyen los términos de la licencia en
inglés al final del mismo.\\

Copyright (c) 2011 David Saltares Márquez.\\

Permission is granted to copy, distribute and/or modify this document under the
terms of the GNU Free Documentation License, Version 1.3 or any later version
published by the Free Software Foundation; with no Invariant Sections, no
Front-Cover Texts, and no Back-Cover Texts. A copy of the license is included in
the section entitled "GNU Free Documentation License".\\

\cleardoublepage

\section*{Notación y formato}

Con el objetivo de mantener un estilo uniforme y cómodamente legible, a lo
largo de esta memoria de \textbf{Proyecto Fin de Carrera} se ha utilizado la
siguiente notación:

\begin{itemize}
    \item Para referirnos a nombres de ficheros, órdenes del sistema o
    funciones de un lenguaje utilizaremos: \texttt{ogre.cfg}.
    \item Cuando se nombre a una tecnología o biblioteca se hará uso del
    formato: \textsc{Ogre3D}.
    \item Los nombres de las clases se escriben en cursiva: \textit{Game}.
    \item En caso de adjuntar un fragmento de código se utilizarán bloques
    como el siguiente:
    \lstinputlisting[style=C++]{codigo/holamundo.cpp}
\end{itemize}
